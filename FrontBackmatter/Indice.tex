\pagestyle{scrheadings}
%\phantomsection
\setcounter{tocdepth}{2} % <-- 2 includes up to subsections in the ToC
\setcounter{secnumdepth}{3} % <-- 3 numbers up to subsubsections
\manualmark
\renewcommand{\contentsname}{Índice general}
\markboth{\spacedlowsmallcaps{\contentsname}}{\spacedlowsmallcaps{\contentsname}}
\tableofcontents
\automark[section]{chapter}
\renewcommand{\chaptermark}[1]{\markboth{\spacedlowsmallcaps{#1}}{\spacedlowsmallcaps{#1}}}
\renewcommand{\sectionmark}[1]{\markright{\thesection\enspace\spacedlowsmallcaps{#1}}}
%*******************************************************
% List of Figures and of the Tables
%*******************************************************
\clearpage
% \pagestyle{empty} % Uncomment this line if your lists should not have any headlines with section name and page number
\begingroup
    \let\clearpage\relax
    \let\cleardoublepage\relax
    %*******************************************************
    % List of Figures
    %*******************************************************
    %\phantomsection
    \renewcommand{\listfigurename}{Listado de figuras}
    \refstepcounter{dummy}
    %\addcontentsline{toc}{chapter}{\listfigurename}
    \pdfbookmark[1]{\listfigurename}{lof}
    
    \listoffigures

    \vspace{8ex}
    \newpage

    %*******************************************************
    % List of Tables
    %*******************************************************
    %\phantomsection
    \renewcommand{\listtablename}{Listado de tablas}
    \refstepcounter{dummy}
    %\addcontentsline{toc}{chapter}{\listtablename}
    \pdfbookmark[1]{\listtablename}{lot}
    \listoftables
    \vspace{8ex}
    \newpage

    %*******************************************************
    % List of Algorithms
    %*******************************************************
    %\phantomsection
    \renewcommand{\algorithmcfname}{Algoritmo}
    \renewcommand{\listalgorithmcfname}{Listado de algoritmos}
    %\refstepcounter{dummy}
    %\addcontentsline{toc}{chapter}{\lstlistlistingname}
    %\pdfbookmark[1]{\listalgorithmcfname}{loa}
    %\lstlistofalgorithms

    %\vspace{8ex}
    %\newpage

    %*******************************************************
    % List of Codes
    %*******************************************************
    %\phantomsection
    \renewcommand{\lstlistingname}{Código}% Listing -> Algorithm
    \renewcommand{\lstlistlistingname}{Listado de Códigos}% List of Listings -> List of Algorithms
    \refstepcounter{dummy}
    %\addcontentsline{toc}{chapter}{\lstlistlistingname}
    \pdfbookmark[1]{\lstlistlistingname}{loa}
    \lstlistoflistings

    \vspace{8ex}
    \newpage

    %*******************************************************
    % Acronyms
    %*******************************************************
    %\phantomsection
    \refstepcounter{dummy}
    \pdfbookmark[1]{Acrónimos}{acronyms}
    \markboth{\spacedlowsmallcaps{Acrónimos}}{\spacedlowsmallcaps{Acrónimos}}
    \chapter*{Acrónimos}
    \begin{acronym}[UMLX]
    	\acro{RRSS}{redes sociales}
        \acro{FOMO}{\textit{fear of missing out}, «temor a dejar pasar» o «temor a perderse algo»}
        \acro{TT}{\textit{trending topic}, «tendencia», «tema de tendencia» o «tema del momento»}
        \acro{PaaS}{\textit{plataforma as a service}, «plataforma como servicio»}
        \acro{YAKE}{\textit{Yet Another Keyword Extractor}}
        \acro{RAKE}{\textit{Rapid Automatic Keyword Extraction}}
        \acro{NLTK}{\textit{Natural Language Toolkit}}
        \acro{VADER}{\textit{Valence Aware Dictionary and sEntiment Reasoner}}
        \acro{NLP}{\textit{Natural Language Processing}, «procesamiento de lenguajes naturales»}
        \acro{DOM}{\textit{Document Object Model}}
        \acro{API}{\textit{application programming interface}, «interfaz de programación de aplicaciones»}
        \acro{REST}{\textit{representational state transfer}, «transferencia de estado representacional»}
        \acro{HTTP}{\textit{Hypertext Transfer Protocol}, «Protocolo de Transferencia de Hipertexto»}
        \acro{JSON}{\textit{JavaScript Object Notation}, «notación de objeto de JavaScript»}
        \acro{YAML}{\textit{YAML Ain't Markup Language}, «YAML no es un lenguaje de marcado»}
        \acro{CRUD}{\textit{Create, Read, Update and Delete}, «Crear, Leer, Actualizar y Borrar»}
        \acro{ASGI}{\textit{Asynchronous Server Gateway Interface}}
        \acro{WSGI}{\textit{Web Server Gateway Interface}}
        \acro{RSS}{\textit{Really Simple Syndication}, «sindicación realmente simple»}
    \end{acronym}

\endgroup 
