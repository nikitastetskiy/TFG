\pdfbookmark[1]{Resumen}{Resumen}
\begingroup

\chapter*{Resumen}

\noindent\spacedlowsmallcaps{Palabras clave:} Aplicación Web, Arquitectura de capas, Modelado API REST, Trending Topics, Noticias, Diseño Responsive.
\begin{center}
· · ·
\end{center}

El principal enfoque de este Trabajo de Fin de Grado se basa en la aportación de información útil, resumida y de fácil alcance para el usuario final, sin que esta sea abrumante y difícil de leer.

\vspace{0.3cm}

A día de hoy, las aplicaciones web sociales más dominantes del mercado intentan captar y mantener al usuario el mayor tiempo posible. Lo consiguen aportando cantidad ingente o casi infinita de información. Esta información se actualiza en cada momento, dando la posibilidad de mantenernos partícipes del mundo social en cada instante.

\vspace{0.3cm}

Dada la atosigante abundancia de información he decidido hacer una aplicación web que proporcione información de una manera mucho más simple sobre tendencias actuales. De esta manera, los usuarios no tendrán que estar tan pendientes en mantenerse actualizados, sino que al final del día tendrán simplificada o resumida la información más destacable y otros datos útiles en base a la tendencia.

\vspace{0.3cm}

La idea que se intenta realizar con este proyecto es una gestión de tendencias a partir de la API de Twitter y la extracción de contenido relevante en base a la tendencia, como puede ser el volumen de publicaciones, palabras claves, sentimientos generales o noticias relevantes.

\vspace{0.3cm}

El proyecto involucra todos los conocimientos, tanto en la ingeniería informática como fuera de ella, que he podido adquirir a lo largo de mi ciclo de aprendizaje del ámbito informático. Además de estudiar la parte programable y funcional de la aplicación, se realiza un estudio riguroso sobre la planificación, estructura y arquitectura ideal al proyecto.

\vspace{0.3cm}

Entrando más en detalle en los conocimientos aplicados, el proyecto se planteará con una arquitectura por capas y un modelo API REST. Los lenguajes de programación usados en su mayoría serán JavaScript, Python y el tipo de bases de datos será NoSQL. 

\vspace{0.3cm}

No obstante, a lo largo del proyecto se estudiará la razón de estas decisiones y se demostrará cómo combinando estas tecnologías, se llega a realizar una única aplicación en cara al usuario final.

\vfill
