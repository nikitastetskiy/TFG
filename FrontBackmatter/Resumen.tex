\pdfbookmark[1]{Resumen}{Resumen}
\begingroup

\chapter*{Resumen}

\noindent\spacedlowsmallcaps{Palabras clave:} Aplicación Web, Arquitectura de capas, Modelado API REST, Trending Topics, Noticias, Diseño Responsive.
\begin{center}
· · ·
\end{center}

El principal enfoque de este Trabajo de Fin de Grado se basa en la aportación de información útil, resumida y de fácil alcance para el usuario final, sin que esta sea abrumante y difícil de leer.

\vspace{0.3cm}

A día de hoy, las aplicaciones web sociales más dominantes del mercado intentan captar y mantener al usuario el mayor tiempo posible. Lo consiguen aportando cantidad ingente o casi infinita de información. Esta información se actualiza en cada momento, dando la posibilidad de mantenernos partícipes del mundo social en cada instante.

\vspace{0.3cm}

Dada la atosigante abundancia de información he decidido hacer una aplicación web que proporcione información de una manera mucho más simple sobre tendencias actuales. De esta manera, los usuarios no tendrán que estar tan pendientes en mantenerse actualizados, sino que al final del día tendrán simplificada o resumida la información más destacable y otros datos útiles en base a la tendencia.

\vspace{0.3cm}

El proyecto involucra todos los conocimientos, tanto en la ingeniería informática como fuera de ella, que he podido adquirir a lo largo de mi ciclo de aprendizaje del ámbito informático.

\vspace{0.3cm}

Entrando más en detalle en los conocimientos aplicados, el proyecto está compuesto por una arquitectura de tres capas: la interfaz de usuario, una capa de gestión y otra de datos. Las tecnologías usadas se pueden resumir en JavaScript, Python y NoSQL. JavaScript, en concreto Vue.js para el apartado Front-end. Para el Back-end, Python como el gestor o el procesador y MongoDB como el sistema de base de datos.

\vspace{0.3cm}

Al combinar estas tecnologías, el modelo que se termina usando es la transferencia de estado representacional o API REST, en el cual se usarán distintos enrutadores y peticiones web, pero que, de cara al usuario final, seguirá siendo una única aplicación.

\vfill
