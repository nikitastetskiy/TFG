\chapter{Análisis}\label{ch:analisis}
%************************************************

\section{Metodología de desarrollo}\label{sec:metodologia}

El proceso de desarrollo de un software puede llegar a ser una tarea muy ardua, durante mucho tiempo esta labor se ha llevado a cabo sin ninguna metodología definida. Para situar el contexto, una metodología se define como una colección de procedimientos, técnicas, herramientas y documentos que llegan a ayudar a los desarrolladores a la hora de implementar software. \cite{gomez2010criterios}

\vspace{0.3cm}

Durante las últimas décadas, se han establecido dos grandes corrientes diferenciales que representan estas metodologías. Por un lado, las metodologías tradicionales se centran en el proceso y controlan estrictamente las actividades que lo acompañan. Las metodologías ágiles, por otro lado, se enfocan en el elemento humano, centrándose en la colaboración y participación del cliente en el proceso a través de iteraciones muy cortas. \cite{gomez2010criterios}

\begin{table}[h]
    \footnotesize
    \centering
    \setlength\arrayrulewidth{0.8pt}
    \begin{tabular}{| >{\centering\arraybackslash}m{2in} | >{\centering\arraybackslash}m{2in} |}

        \hline
        \rowcolor{RoyalBlue}
        \textbf{Metodologías Ágiles} & \textbf{Metodologías Tradicionales} \\
        \hline
        Basadas en heurísticas provenientes de prácticas de producción de código & Basadas en normas provenientes de estándares seguidos por el entorno de desarrollo \\
        \hline
        Especialmente preparados para cambios durante el proyecto & Cierta resistencia a los cambios \\
        \hline
        Impuestas internamente (por el equipo) & Impuestas externamente \\
        \hline
        Proceso menos controlado, con pocos principios & Proceso mucho más controlado, con numerosas políticas/normas \\
        \hline
        No existe contrato tradicional o al menos es bastante flexible & Existe un contrato prefijado \\
        \hline
        El cliente es parte del equipo de desarrollo & El cliente interactúa con el equipo de desarrollo mediante reuniones \\
        \hline
        Grupos pequeños (menos de 10 integrantes) y trabajando en el mismo sitio & Grupos grandes y posiblemente distribuidos \\
        \hline
        Pocos artefactos y pocos roles & Más artefactos y más roles \\
        \hline
        Menos énfasis en la arquitectura del software & La arquitectura del software es esencial y se expresa mediante modelos \\
        \hline
        
    \end{tabular}

    \caption[Comparación de metodologías]{Comparación de metodologías. \cite{canos2003metodologias}}\label{table:comparacion_metodologias}

\end{table}

La parte más importante de las fases iniciales es poder determinar qué metodología es mejor para nuestro proyecto con el fin de lograr los mejores resultados de manera oportuna.

\vspace{0.3cm}

El objetivo general de implementar una metodología es llegar a construir un producto de alta calidad. Esta elección implica un conjunto de principios básicos que deben ser seguidos y respetados. Estos incluyen actividades claras para comprender el problema y comunicarse con los clientes, un método definido para representar el diseño, las mejores prácticas para implementar la solución y estrategias sólidas. \cite{maida2015metodologias}

\vspace{0.3cm}

En base a las características mencionadas y lo explicado en la sección de planificación, se ha considerado que la solución óptima sería una metodología de desarrollo ágil. Considerando que el problema puede cambiar a lo largo del proyecto, por lo que necesitamos una metodología adaptativa y flexible, además la forma de trabajar concuerda mucho más (la cantidad del grupo o su propio manejo).

\subsection{Metodología de trabajo escogida (SCRUM)}
El nacimiento de la idea de la metodología Scrum fue en el año 1986, con el propósito de aumentar la velocidad de desarrollo de una aplicación y a la vez conservando flexibilidad. Los japonenses Hirotaka Takeuchi e Ikujiro Nonaka  se inspiraron en el concepto de un equipo de rugby de 15 jugadores unido y con el mismo objetivo. Durante los años posteriores este concepto fue remodelado y reforzado hasta convertirse en lo que es a día de hoy, el método ágil más usado en el mundo. \cite{rodriguez2015que}

\vspace{0.3cm}

Scrum puede llegar a ser ventajoso debido a que el cliente se siente comprometido con el proyecto, implicándose en distintas necesidades de tipo funcional y permitiendo realizar revisiones tempranas de los desarrollos fomentando de esta manera las propias entregas. Otras ventajas a destacar pueden llegar a ser su adaptación o trabajo en equipo. \cite{rodriguez2015que}

\vspace{0.3cm}

Los componentes de la metodología Scrum son: \cite{rad2019fundamentos}

\begin{enumerate}

    \item
    Los \textbf{Roles de Scrum}: son los papeles o funciones de cada integrante del equipo, que puede ser:
    
        \begin{itemize}
            \item
            \textit{El Dueño del Producto (Product Owner)}: representará la función del cliente, con el objetivo de definir historias de usuario y validar la calidad del producto.
    
            \item
            \textit{Scrum Master}: su función es velar por la actividad o ejecución del Scrum, para que se realice correctamente por el equipo. También será el responsable de dirigir y concretar los diferentes eventos.
    
            \item
            \textit{El Equipo de Desarrollo}: son los que tienen la función de realizar y concretar las historias de usuario.
    
            \item
            \textit{Stakeholders}: no tienen función en la ejecución o realización del trabajo, pero están interesados en el producto y represan un papel de suma importancia.
    
        \end{itemize}

    \item
    Los \textbf{Eventos Scrum}: son las reuniones que se llevan a cabo en el equipo de trabajo. Están diseñados para la comunicación y colaboración continua entre equipos, con el objetivo de minimizar la necesidad de mantener reuniones indefinidas que pueden afectar de alguna manera a la planificación del proyecto. Se subdividen en estos tipos:

        \begin{itemize}
            \item
            \textit{El Sprint}: trata de un tramo de tiempo fijo, que puede llegar a ser entre una y cuatro semanas, durante el cual el producto que se esté desarrollando añade valor gradualmente, de manera que al final de cada sprint el cliente pueda tener una clara visión del producto.
    
            \item
            \textit{La planificación del Sprint (Sprint Planning)}: es una reunión que se lleva a cabo al comienzo o inicio de cada sprint para planificar el trabajo que se realizará a lo largo del sprint.
    
            \item
            \textit{Scrum Diario (Dailys)}: reuniones diarias de corta duración, en estas reuniones los miembros del equipo discuten y dialogan el trabajo realizado en el día anterior y el trabajo que se realizará en el día de la reunión.
    
            \item
            \textit{Revisión del Sprint (Sprint Review)}: una reunión donde cada miembro del equipo presenta lo que ha realizado a lo largo del sprint y lo que debería cambiar o mejorar en el producto para el próximo sprint.

            \item
            \textit{Retrospectiva del Sprint (Sprint Retrospective)}: reuniones con la intención de analizar el trabajo en equipo y discutir áreas para mejorar o colaborar como equipo para el próximo sprint.
    
        \end{itemize}

    \item
    Los \textbf{Artefactos de Scrum}: utilizadas para aumentar la transparencia de la información para que todos los miembros del equipo tengan constancia sobre lo que está ocurriendo. Se subdividen en:
    
        \begin{itemize}
            \item
            \textit{El Backlog del Producto (Product Backlog)}: contienen todas las historias de usuario, pero no evalúan el proyecto, y su gestión o responsabilidad recae en el Dueño del Producto.
    
            \item
            \textit{El Backlog del Sprint (Sprint Backlog)}: historias de usuario que contienen estimaciones y planes a realizar en el sprint.
    
            \item
            \textit{Incremento}: es el producto que se produce o ocurre al final del sprint.
    
        \end{itemize}

\end{enumerate}

\vspace{0.3cm}

\subsection{Adaptación de la metodología de trabajo escogida}
Habiendo definido los componentes en la sección anterior, es visible que Scrum está dirigido para trabajar en equipo, por lo que se tendrá que adaptar esta metodología al entorno académico del proyecto, que en el caso de este proyecto se reduciría a mi persona, Nikita Stetskiy. Debido a esto, me veré obligado a representar todos los roles del equipo y mis tutores darán función a los stakeholders y a Scrum Master.

\vspace{0.3cm}

Además, se puede dividir el proyecto en dos productos. La memoria, donde el cliente será el tribunal del trabajo de fin de grado y los tutores serán los Dueños del Producto. Mientras que para la propia aplicación, tendremos un cliente ficticio, por ejemplo Mauricio Mares, y el Dueño del Producto seré yo mismo, puesto que seré el que mejor conoceré los intereses del cliente.

\vspace{0.3cm}

En cuanto a los eventos, el Scrum diario no llega a tener sentido en este entorno, al estar formado por solo una persona y los demás eventos se pueden juntar en un mismo encuentro e iteración.

\vspace{0.3cm}

Finalmente, respecto a la velocidad del proyecto, se llegó a detallar en la sección \ref{ch:planificacion_detallada} de la Planificación. Cada Sprint tiene acordada una duración de dos semanas.

\newpage
\section{Análisis del entorno}

Es necesario estudiar el entorno y a los posibles usuarios que necesiten de nuestro servicio antes de empezar a desarrollar el \textit{Backlog} del Producto. Para ello propondré dos diferentes casos ficticios que estudiaremos a continuación.

\begin{table}[h]
\centering
    \begin{tabular}{| >{\centering\arraybackslash}m{0.75in} >{\centering\arraybackslash}m{3.5in}|}
        \hline
\multicolumn{2}{|c|}{\cellcolor{RoyalBlue}\textbf{1ª Persona}}                                                                                                                                                                                                                                                                                                                                                                                                                      \\ \hline
\multicolumn{1}{|c|}{\textbf{Nombre}}                                                                                                                                                                                                                  & Mauricio Mares                                                                                                                                                                                                                \\ \hline
\multicolumn{1}{|c|}{\textbf{Edad}}                                                                                                                                                                                                                    & 23                                                                                                                                                                                                                            \\ \hline
\multicolumn{1}{|c|}{\textbf{Sexo}}                                                                                                                                                                                                                    & Masculino                                                                                                                                                                                                                     \\ \hline
\multicolumn{2}{|c|}{\textbf{Datos relevantes}}                                                                                                                                                                                                                                                                                                                                                                                                                                        \\ \hline
\multicolumn{2}{|p{4.25in}|}{Terminó la carrera de bellas artes, aunque ahora mismo no se siente lo suficientemente apasionado para trabajar de ello. Uno de los principales motivos es el sobre uso de las redes sociales en su día a día.}                                                                                                                                                                                                                                                    \\ \hline
\multicolumn{2}{|c|}{\textbf{Contexto de uso}}                                                                                                                                                                                                                                                                                                                                                                                                                                         \\ \hline
\multicolumn{2}{|p{4.25in}|}{El uso que pretende dar este usuario corresponde a un ámbito casual, ya que pretende usar menos las redes sociales, pero a la par quiere mantenerse actualizado de lo que pasa en el mundo. Por lo que la función que requiere de la aplicación es de informarse de eventos importantes que transcurran diariamente o incluso semanalmente.}                                                                                                                                                         \\ \hline
\multicolumn{2}{|c|}{\textbf{Necesidad diferencial}}                                                                                                                                                                                                                                                                                                                                                                                                                                   \\ \hline
\multicolumn{2}{|p{4.25in}|}{La funcionalidad a destacar es que la aplicación deberá guardar todas las tendencias del día y poder dar datos que resuman información de dichas tendencias. Además, Mauricio no tiene que estar constantemente atento a la aparición de una nueva tendencia, ya que se van guardando por días. Esto le permitirá a Mauricio reducir las horas de consumo diario de aplicaciones que requerían su atención constante a cambio de información relevante o actual.} \\ \hline
\end{tabular}
\caption[Primer caso de estudio de entorno]{Primer caso de estudio de entorno.}\label{table:1-entorno-persona}
\end{table}

En este primer caso necesitamos que la aplicación sea capaz de resumir información de tendencias diarias y que a la vez no sea atosigante o que agobie con demasiada cantidad de información, por ende, tiene que ser información simple y finita. La posible solución más óptima es organizar tendencias diarias en un \textit{top} o una clasificación de diez tendencias más populares ocurridas a lo largo del día. Además, esto se puede guardar por semanas para que el usuario no tenga la obligación de entrar a la web diariamente.

\vspace{0.3cm}

El carácter de la información, simplificada para el usuario que tendrán las tendencias, será puramente informativo y además intentará ser objetiva. La principal diferencia es que al usar las \ac{RRSS}, la información relevante se consigue por medio de la atención. Cuantas más publicaciones u opiniones de personas se lean, más informado estará el usuario. Por lo que el factor diferencial, es que la página web en desarrollo mostrará datos simplificados o resumidos de manera objetiva. Todos estos datos estarán comprimidos de tal manera, que el usuario pueda tener una visión clara sobre la tendencia sin la necesidad de leer un alto número de publicaciones u opiniones.

\vspace{0.3cm}

La información que se aporta al usuario en la aplicación web será modular. De tal manera que cada módulo, que tendrá el diseño de una carta, informará de un dato de interés. Los datos más importantes que puede tener una tendencia y que sea útil para el usuario son datos como la relevancia que está teniendo la tendencia, las palabras o \textit{keywords} más usadas, las opiniones de las personas (vistas de manera objetiva a través de un análisis de sentimientos) y diferentes noticias actuales relacionadas con la propia tendencia.

\vspace{0.3cm}

De esta manera, al aportar información de manera simplificada, el usuario Mauricio podrá informarse diariamente de manera adecuada y sin la necesidad de perder el tiempo en las \ac{RRSS}.


\begin{table}[h]
\centering
    \begin{tabular}{| >{\centering\arraybackslash}m{0.75in} >{\centering\arraybackslash}m{3.5in}|}
        \hline
\multicolumn{2}{|c|}{\cellcolor{RoyalBlue}\textbf{2ª Persona}}                                                                                                                                                                                                                                                                                                                                                                                                                      \\ \hline
\multicolumn{1}{|c|}{\textbf{Nombre}}                                                                                                                                                                                                                  & Nuria Nevadas                                                                                                                                                                                                                \\ \hline
\multicolumn{1}{|c|}{\textbf{Edad}}                                                                                                                                                                                                                    & 37                                                                                                                                                                                                                            \\ \hline
\multicolumn{1}{|c|}{\textbf{Sexo}}                                                                                                                                                                                                                    & Femenino                                                                                                                                                                                                                     \\ \hline
\multicolumn{2}{|c|}{\textbf{Datos relevantes}}                                                                                                                                                                                                                                                                                                                                                                                                                                        \\ \hline
\multicolumn{2}{|p{4.25in}|}{Terminó hace mucho el grado de Empresariales, en los últimos meses empezó a invertir en Bolsa, por su propia cuenta de manera profesional.}                                                                                                                                                                                                                                                    \\ \hline
\multicolumn{2}{|c|}{\textbf{Contexto de uso}}                                                                                                                                                                                                                                                                                                                                                                                                                                         \\ \hline
\multicolumn{2}{|p{4.25in}|}{El uso que pretende dar este usuario corresponde a un ámbito profesional. Pretende tener una visión pequeña y objetiva de búsquedas concretas, en este caso del mercado financiero.}                                                                                                                                                         \\ \hline
\multicolumn{2}{|c|}{\textbf{Necesidad diferencial}}                                                                                                                                                                                                                                                                                                                                                                                                                                   \\ \hline
\multicolumn{2}{|p{4.25in}|}{La funcionalidad a destacar en esta aplicación es que se podrá buscar tópicos y obtener la misma información actual que se obtienen también con las tendencias, es decir, datos de interés objetivos.} \\ \hline
\end{tabular}
\caption[Segundo caso de estudio de entorno]{Segundo caso de estudio de entorno.}\label{table:1-entorno-persona}
\end{table}

Al estudiar el segundo caso, la solución que se le plantea al usuario llega a ser bastante parecida al primer caso. El principal factor diferencial es que la información que se aporta en el primer caso es sobre las tendencias y en este caso tendrá que ser sobre búsquedas específicas.

\vspace{0.3cm}

No se podrá dar información sobre la popularidad del tópico, ya que no es una tendencia, pero si información sobre las \textit{keywords} o palabras más usadas, el sentimiento general y las noticias más destacables. Esta información tendrá que ser recopilada de distintas publicaciones que tengan relevancia con el tópico.

\subsection{Estudio competitivo}

En la actualidad las principales aplicaciones que existen que recopilan tendencias a diario son Google Trends y parcialmente la plataforma de Twitter. Ambas poseen fallos, el principal problema de Twitter se incluye en los riesgos de un uso continuo de una \ac{RRSS} que se explicó anteriormente en la sección \ref{ch:estado-del-arte}. Por otro lado, Google Trends posee multiples problemas a la hora de afrontar nuestras necesidades. Primero, no funciona en todos los países, incluyendo España. Además, no contienen información relevante que pueda resumir una tendencia o el porqué de su relevancia. Aunque Google Trends posea un exhaustivo historial de las búsquedas que ha tenido dicha tendencia, posibles noticias de interés y tópicos similares, no posee el sentimiento general que está teniendo dicha tendencia y tampoco diferentes \textit{keywords} en relación (esto debido a que carece de publicaciones como Twitter). En resumen, ambas aplicaciones principales, tienden a ser incompletas a nuestras necesidades.

\vspace{0.3cm}

También existen otras páginas que usan información de Twitter o Google. Al buscar páginas o aplicaciones web sobre la información de tendencias actuales de Twitter salen resultados como Trendinalia, GetDayTrends o Trends24. Este tipo de páginas web llega a mostrar un contenido puramente estadístico sobre las tendencias, con el uso exclusivo de diagramas o distribuciones globales para explicar el comportamiento de una tendencia.

\vspace{0.3cm}

Realmente este tipo de páginas no indagan en la explicación del contenido de una tendencia para que el usuario pueda estar informado. El objetivo de estas páginas es informar al usuario de datos numéricos que se calculan mientras la tendencia exista, como puede ser la posición de la tendencia respecto a otras durante las últimas 24 horas.

\vspace{0.3cm}

Esto puede llegar a ser información útil si solo quieres saber datos sobre el comportamiento de la tendencia a lo largo de 24 horas. En el caso de nuestra aplicación, aparte de dar datos sobre el comportamiento de la tendencia, tendrá el objetivo de explicar el propósito de la tendencia.

\subsection{Estudio competitivo técnico}\label{sub:estudio-competitivo-tecnico}

Actualmente las librerías más populares sobre recopilación de tendencias actuales son Tweepy y Pytrends, ambas programadas en el lenguaje Python. Dichas librerías distan de ser perfectas, pero cada una tiene características de carácter diferencial.

\vspace{0.3cm}

\begin{itemize}
    \item
    \textbf{Pytrends} \cite{pytrends-manual}
    
    Esta librería es una API no oficial que usa información de Google Trends. Por lo que es una librería que aporta información muy valiosa respecto a diferentes tendencias. Esta librería posee numerosas ventajas, a la par que desventajas. Las siguientes características son los métodos que tiene la librería disponible:
    \begin{enumerate}
    \item \textit{Interés a lo largo del tiempo}: devuelve datos históricos e indexados de las búsquedas sobre una palabra clave. Esta información se muestra en la sección Interés a lo largo del tiempo de Google Trends.
    \item \textit{Interés histórico por hora}: devuelve datos históricos, indexados y por hora de las búsquedas sobre una palabra clave. Envía múltiples solicitudes a Google, cada una de las cuales recupera una semana de datos por hora.
    \item \textit{Interés por región}: devuelve datos sobre el lugar en el que se busca la palabra clave. Información mostrada en la sección Interés por región de Google Trends.
    \item \textit{Temas relacionados}: devuelve datos de las palabras clave relacionadas con una palabra clave proporcionada. Se muestra en la sección Temas relacionados de Google Trends.
    \item \textit{Consultas relacionadas}: devuelve datos de las palabras clave relacionadas con una palabra clave proporcionada. Se muestra en la sección Consultas relacionadas de Google Trends.
    \item \textit{Búsquedas de tendencias}: devuelve datos de las búsquedas de tendencias más recientes. Se muestran en la sección Búsquedas de tendencias de Google Trends.
    \item \textit{Gráficos principales}: devuelve los datos de un tema determinado. Se muestra en la sección Gráficos principales de Google Trends.
    \item \textit{Sugerencia}: devuelve una lista de palabras clave sugeridas adicionales que se pueden usar para refinar una búsqueda de tendencia.
    \end{enumerate}

    Estas características pueden llegar a ser extremadamente útiles a la hora de gestionar nuestra aplicación. Aunque las principales desventajas son demasiado grandes como para no tenerlas en cuenta:

    \begin{enumerate}
    \item \textit{Disponibilidad geográfica}: debido a ciertas leyes y normativas, Google Trends no funciona en todos los países del mundo. Por desgracia, España está excluida de momento, aunque se tiene pensado cambiar la normativa o adaptarse a ella.
    \item \textit{Longevidad}: esta API al no ser oficial, no tiene garantía de estar funcionando siempre. Ya que llegará un momento donde se cambie el \textit{Back-End} de Google e incidirá en el comportamiento de esta API.
    \item \textit{Limitaciones de llamadas}: El límite de peticiones no se conoce públicamente. Aunque los usuarios hicieron medidas aproximadas, donde se demostraba que 1400 solicitudes secuenciales en un período de tiempo de 4 horas daban lugar al límite (replicado en 2 redes), aunque se podía volver a hacer peticiones después de 60 segundos de tiempo de espera.
    \end{enumerate}

\end{itemize}

\begin{itemize}
    \item
    \textbf{Tweepy} \cite{tweepy-manual}
    
    Es un paquete de código abierto que brinda una forma muy conveniente de acceder a la API de Twitter con Python. Tweepy incluye un conjunto de clases y métodos que representan los modelos de Twitter y los extremos de la API. Las principales características para nuestro uso son las siguientes:
    \begin{enumerate}
    \item \textit{Obtención de tendencias cerca de una ubicación}: devuelve las 50 tendencias principales para un WOEID (código de país) específico, si la información de tendencias está disponible. La respuesta es una lista de objetos "trend" que codifican el nombre del tema de tendencia, el parámetro de consulta que se puede usar para buscar el tema en la búsqueda de Twitter y la URL de búsqueda de Twitter. Esta información se almacena en caché durante 5 minutos.
    \item \textit{Búsqueda de publicaciones}: devuelve una colección de publicaciones o \textit{tweets} relevantes que coinciden con una consulta específica. Aunque la API de búsqueda no pretende ser una fuente exhaustiva de Tweets. No todos los Tweets se indexarán o estarán disponibles a través de la interfaz de búsqueda.
    \end{enumerate}

    Estas características pueden llegar a cumplir las necesidades explicadas anteriormente, aunque también tenemos que tener en cuenta las desventajas que presenta esta API.

    \begin{enumerate}
    \item \textit{Disponibilidad de versiones}: existen actualmente varias versiones de la API publicadas, cada una se accede de una manera diferente y tiene métodos diferentes. La documentación oficial de Twitter no está bien cuidada y no funciona de la misma manera que la aplicación web. Un sencillo ejemplo de ello es que en la web se puede buscar tweets de un país en concreto, mientras que esta función no existe en la API.
    \item \textit{Historial pobre}: mientras que Google guarda todos los datos de interés sobre una tendencia, Twitter no, por lo que búsquedas antiguas no contendrán mucho valor al usuario en cuanto a la relevancia o interés que presenta un tópico.
    \end{enumerate}

\end{itemize}

\subsection{Conclusiones de los estudios realizados}

En los apartados anteriores se definieron claramente las ventajas y desventajas del uso de ambas aplicaciones o APIs. Se puede concluir que, aunque Google Trends posea más funciones versátiles y con mayor peso de utilidad en cuanto a nuestras necesidades, sigue teniendo mayores desventajas a la hora de usarlo. La mayor de todas es que no esté disponible en todos los países. La plataforma de twitter en cambio se usa en casi todo el mundo, además se pueden llegar a implementar funciones parecidas con métodos propios y no tener que depender en un mayor nivel de una API externa.

\vspace{0.3cm}

En cuanto a los aspectos que se deberán tratar a la hora de definir las características o funcionalidades de la aplicación se definirán en la siguiente lista. Aunque se entrará más en detalle en la sección posterior.

\begin{itemize}
    \item
    Las personas requieren información relevante, concisa y simplificada.
    \item
    La información debe ser de contenido finito, a la misma hora aportando al usuario su pleno entendimiento.
    \item
    Los usuarios deberán ser capaces de usar la aplicación en todos los dispositivos que tengan acceso web. Además, deberá tener un diseño \textit{responsive}, para que se adapte a la interfaz del usuario.
    \item
    Los usuarios tienen que ser capaces de navegar a través de distintas tendencias. Poder elegir tanto el país de búsqueda como la fecha para poder informarse debidamente.
    \item
    Los usuarios tienen que ser capaces de realizar búsquedas de distintos tópicos de interés y poder informarse con datos relacionados y actuales.
    \item
    El diseño será presentado mediante módulos o cartas. Cada carta tendrá información útil y de necesidad al usuario.
    \item
    Un módulo contendrá datos sobre la popularidad de la tendencia y su correspondiente gráfico.
    \item
    Otro módulo contendrá datos sobre palabras y \textit{keywords} más usadas sobre la tendencia y su correspondiente gráfico.
    \item
    Otro módulo contendrá un sentimiento general y un gráfico visual sobre dicho comportamiento.
    \item
    También habrá tres módulos de noticias de interés general sobre la tendencia.
    \item
    Cada módulo deberá respetar tanto el idioma como la localización elegida por el usuario.
    \item
    Se deberá tener en cuenta que el diseño sea simple y agradable al usuario.
    \item
    Otro factor a tener en cuenta es el plan de peticiones que ofrece Twitter, el cual es muy limitado.
\end{itemize}

\newpage

\section{Requisitos del sistema}
Gracias a los estudios y conclusiones realizadas en el capítulo anterior, se puede detallar de una manera más sencilla los siguientes requerimientos o requisitos. Los requerimientos funcionales funcionan como servicios que puede prestar el sistema. Mientras que los requerimientos no funcionales especifican más bien la fiabilidad o el comportamiento de este.

\subsection{Requisitos Funcionales}

\begin{enumerate}
    { \renewcommand\labelenumi{R.F. \theenumi}
    \item
    \textit{Gestión de Trends}. El sistema deberá ser capaz de realizar llamadas al servicio de la API de Twitter para recopilar tendencias. En la API o su plataforma se denominan Trends o \ac{TT}.
        \begin{enumerate}{\renewcommand{\labelenumii}{R.F. \arabic{enumi}.\arabic{enumii}}
        \item
        \textit{Gestión de Países}. El sistema deberá ser capaz de manejar la petición concorde a diferentes países y almacenarlas adecuadamente.
            \begin{enumerate}{\renewcommand{\labelenumiii}{R.F. \arabic{enumi}.\arabic{enumii}.\arabic{enumiii}}
            \item
            \textit{Búsqueda por WOEID}. El sistema deberá ser capaz de buscar los diferentes países por el identificador que provee Yahoo y que utiliza la plataforma de Twitter. En este caso se denominan WOEIDs.
            }\end{enumerate}
        \item
        \textit{Gestión de la Fecha}. El sistema deberá ser capaz de manejar la respuesta generada por Twitter y poder diferenciarla por fecha, bajo el formato de yy-mm-dd.
        \item
        \textit{Limpieza de Trends}. El sistema deberá poder limpiar los Trends repetidos. Ya que la mayoría de las veces Twitter no reconoce el mismo \ac{TT}, casos como el uso de tilde, mayúsculas o si son una sucesión de varias palabras. Por ejemplo, «Balón», «balón», «balon» y «Balón de Oro» se referirán a la misma tendencia, pero Twitter los trata como Trends diferentes con datos respectivos diferentes.
        \begin{enumerate}{\renewcommand{\labelenumiii}{R.F. \arabic{enumi}.\arabic{enumii}.\arabic{enumiii}}
            \item
            \textit{Preferencias de Limpieza}. El sistema preferirá un nombre de tendencias más largo y por defecto si son iguales, preferirá el que tenga más popularidad.
            }\end{enumerate}
        \item
        \textit{Ordenación de los Trends}. El sistema deberá ser capaz de ordenar los \ac{TT} bajo el atributo del volumen de popularidad proporcionado por Twitter.
        \item
        \textit{Historial Inteligente}. Además de guardar los Trends ordenados, al usuario le será importante que se guarden y se vayan comparando los Trends viejos y los nuevos (mediante el atributo de la popularidad).
        }\end{enumerate}
    \item
    \textit{Gestión de Popularidad}. El sistema deberá ser capaz de guardar el atributo del volumen de popularidad en una lista, que fue anteriormente proporcionado por Twitter.
    \begin{enumerate}{\renewcommand{\labelenumii}{R.F. \arabic{enumi}.\arabic{enumii}}
        \item
        \textit{Cálculo de Media}. El sistema deberá poder sacar la media o la popularidad promedio a partir de la lista guardada anteriormente.
        \item
        \textit{Cálculo de Moda}. El sistema deberá poder sacar la moda o la popularidad más grande a partir de la lista guardada anteriormente.
        \item
        \textit{Gestión de Hora}. El sistema tiene que guardar en una lista la hora a la que se ha calculado el valor de la popularidad.
        \item
        \textit{Gestión de Gráfico}. Se deberá proporcionar un gráfico a partir de los valores calculados anteriormente. Proporcionado en formato de área.
        \begin{enumerate}{\renewcommand{\labelenumiii}{R.F. \arabic{enumi}.\arabic{enumii}.\arabic{enumiii}}
            \item
            \textit{Eje X}. Se deberá proporcionar la lista de la hora a la que se ha calculado la popularidad como eje X del gráfico.
            \item
            \textit{Eje Y}. Se deberá proporcionar la lista del vector de popularidad como el eje Y del gráfico.
            \item
            \textit{Análisis del Gráfico}. Además del gráfico visual, se deberán proporcionar datos de valor. En este caso, el primer dato calculado y el último, además de mostrar la hora de su cálculo.
            }\end{enumerate}
        }\end{enumerate}
    \item
    \textit{Gestión de Tweets}. El sistema deberá ser capaz de guardar como máximo alrededor de cien publicaciones o tweets de usuarios diferentes.
    \begin{enumerate}{\renewcommand{\labelenumii}{R.F. \arabic{enumi}.\arabic{enumii}}
        \item
        \textit{Filtrado de Tweets}. Se debe filtrar los tweets mediante la API de Twitter. Primero que las publicaciones tengan un nº mínimo de interacciones y que a la vez sean recientes, luego por el lenguaje del país y que sean publicaciones originales no repetidas.
        \item
        \textit{Limpieza de Tweets}. Se debe limpiar los tweets como los nombres de usuarios, enlaces u otros caracteres que complicarían el análisis del texto.
        \begin{enumerate}{\renewcommand{\labelenumiii}{R.F. \arabic{enumi}.\arabic{enumii}.\arabic{enumiii}}
            \item
            \textit{Gestión de Stop Words}. El sistema deberá diferenciar las palabras significativas de las que no, frecuentemente denominadas como \textit{stop words} también llamadas palabras vacías o palabras comunes, son palabras que no suelen aportar significado a la hora de analizar textos. Cada idioma tiene su propia lista de \textit{stop words}.
            }\end{enumerate}
        }\end{enumerate}
        \item
        \textit{Clasificación de Palabras y Gestión de Keywords}. Al tener un conjunto de palabras significativas, el sistema deberá disponerlas al usuario como un top o clasificación ordenadas por el número de repeticiones en el texto. Por otro lado, el sistema deberá extraer \textit{keywords} relevantes del conjunto de tweets mediante un algoritmo de extracción.
        \begin{enumerate}{\renewcommand{\labelenumii}{R.F. \arabic{enumi}.\arabic{enumii}}
            \item
            \textit{Filtrado de Keywords}. Se deberán eliminar \textit{keywords} parecidas o repetidas.
            \item
            \textit{Gestión de Gráfico}. Se deberá proporcionar un gráfico a partir de los valores calculados anteriormente. Mostrado como un gráfico de tipo barras radial.
            \begin{enumerate}{\renewcommand{\labelenumiii}{R.F. \arabic{enumi}.\arabic{enumii}.\arabic{enumiii}}
                \item
                \textit{Conjunto de Datos}. El conjunto de datos será proporcionado por la clasificación de seis palabras más repetidas de los tweets recogidos. Además, deberán representarse como un porcentaje.
                }\end{enumerate}
            }\end{enumerate}
        \item
        \textit{Gestión de Sentimiento General}. Al tener un conjunto de palabras significativas, el sistema deberá disponerlas al usuario como un sentimiento general, clasificando el conjunto de palabras como positivo, negativo o neutral.
        \begin{enumerate}{\renewcommand{\labelenumii}{R.F. \arabic{enumi}.\arabic{enumii}}
            \item
            \textit{Gestión de Gráfico}. Se deberá proporcionar un gráfico a partir de los valores calculados anteriormente. Mostrado como un gráfico de tipo burbujas con eje X y eje Y.
            \begin{enumerate}{\renewcommand{\labelenumiii}{R.F. \arabic{enumi}.\arabic{enumii}.\arabic{enumiii}}
                \item
                \textit{Eje X}. Este eje representará el sentimiento general de las publicaciones o tweets.
                \item
                \textit{Eje Y}. Este eje representará la relación con la clasificación de palabras, calculadas anteriormente, de las publicaciones o tweets.
                \item
                \textit{Tamaño de las Burbujas}. Se representará un tamaño mayor o menor, en base a la popularidad de los tweets o las publicaciones.
                }\end{enumerate}
            }\end{enumerate}
        \item
        \textit{Gestión de Noticias}. Se deberán proporcionar tres noticias adecuadas a la tendencia. Se buscarán noticias parecidas gracias al nombre y las \textit{keywords} relacionadas a la tendencia.
        \item
        \textit{Gestión de Rutas}. El sistema deberá disponer de rutas de navegación por cada país y fecha correspondiente a las tendencias.
        \item
        \textit{Sistema de Búsqueda}. El sistema deberá disponer de un sistema de búsqueda que proporcione tópicos y datos. 
        \item
        \textit{Gestión de Guardado}. El sistema deberá poder guardar las tendencias y que el usuario pueda acceder a diferentes países o fechas.
}\end{enumerate}

\subsection{Requisitos No Funcionales}

\begin{enumerate}
    { \renewcommand\labelenumi{R.N. \theenumi}
    \item
    \textit{Diseño}. El diseño debe ser sencillo, minimalista, pero a la vez poder dar toda la información que el usuario requiera. Además, debe ser \textit{responsive}, enfocada en el uso móvil.
    \item
    \textit{Disponibilidad}. El sistema debe intentará estar disponible siempre y actualizar el contenido relevante cada hora. Confiando en \ac{PaaS} a la hora de construir, correr y mantener la aplicación.
    \item
    \textit{Arquitectura de capas}. El sistema deberá tener una base de datos, una capa de negociación o gestión y por último una capa de presentación o interfaz.
    \item
    \textit{Sistema de Logs}. El sistema deberá tener un sistema de Logs, el cual recopile información fundamental o registros de la actividad del servidor.
    \item
    \textit{Tests}. El sistema deberá estar \textit{testeado} para poder comprobar su correcto funcionamiento. Además, tendrá reglas de escritura de código específicas, considerado como buenas prácticas para dicho lenguaje de programación.
}\end{enumerate}

\newpage

\section{Product Backlog}
Después de listar los distintos requisitos, podemos visualizar más fácilmente la lista de Historias de Usuario que se trabajará y desarrollará posteriormente. La siguiente tabla contiene las diferentes Historias de Usuario, donde la estimación del esfuerzo (E) está expresada mediante los Puntos de Historia (1 a 10) y la prioridad (P) representada del 1 a 3, siendo el 1 el más prioritario para el Dueño del Producto, es decir, los tutores.

\begin{table}[h]
\centering
\small
\begin{tabular}{| >{\centering\arraybackslash}m{0.55in} | >{\centering\arraybackslash}m{3in} | >{\centering\arraybackslash}m{0.1in} | >{\centering\arraybackslash}m{0.1in} |}
\hline
\rowcolor{RoyalBlue} 
\textbf{ID} & \textbf{Título de la Historia} & \textbf{E} & \textbf{P} \\ \hline
H.U. 1  & \multicolumn{1}{p{3in}|}{Cualquier usuario puede usar la aplicación sin necesidad de registrarse.}   & 1 & 1  \\ \hline
H.U. 2  & \multicolumn{1}{p{3in}|}{El usuario debe poder visualizar las diez tendencias más populares por defecto.} & 5  & 1  \\ \hline
H.U. 3  & \multicolumn{1}{p{3in}|}{El usuario debe poder visualizar las diez tendencias más populares, pudiendo seleccionar el país como parámetro.} & 5  & 1  \\ \hline
H.U. 4  & \multicolumn{1}{p{3in}|}{El usuario debe poder visualizar las diez tendencias más populares, pudiendo seleccionar la fecha como parámetro.} & 5  & 1  \\ \hline
H.U. 5  & \multicolumn{1}{p{3in}|}{El usuario puede buscar sus propios tópicos y visualizarlos del mismo modo que las tendencias.} & 4  & 2  \\ \hline
H.U. 6  & \multicolumn{1}{p{3in}|}{El usuario debe saber, por medio de alertas, si el servicio o los parámetros no han funcionado.} & 3  & 3  \\ \hline
H.U. 7  & \multicolumn{1}{p{3in}|}{El usuario tiene que poder navegar intuitivamente, es decir por medio de gestos de \textit{scroll}, por la página.} & 7  & 1  \\ \hline
H.U. 8  & \multicolumn{1}{p{3in}|}{El usuario debe poder analizar datos de interés sobre la popularidad.} & 8  & 1  \\ \hline
H.U. 9  & \multicolumn{1}{p{3in}|}{El usuario debe poder visualizar las palabras más repetidas o comunes mediante un porcentaje, a partir de los tweets recogidos referentes a la tendencia.} & 8  & 1  \\ \hline
H.U. 10  & \multicolumn{1}{p{3in}|}{El usuario debe poder visualizar las \textit{keywords}, a partir de los tweets recogidos referentes a la tendencia.} & 8  & 1  \\ \hline
H.U. 11  & \multicolumn{1}{p{3in}|}{El usuario puede analizar los sentimientos generales mediante un porcentaje mostrado.} & 9  & 2  \\ \hline
H.U. 12  & \multicolumn{1}{p{3in}|}{El usuario debe poder ver tres noticias más actuales relacionadas con la tendencia y poder acceder a ellas.} & 10  & 1  \\ \hline
\end{tabular}
\caption[Títulos de Historias de Usuario]{Títulos de Historias de Usuario.}
\end{table}

\newpage

Además, algunas Historias de Usuario pueden tener diferentes subniveles, como se detalla en la siguiente tabla.

\begin{table}[h]
\centering
\small
\begin{tabular}{| >{\centering\arraybackslash}m{0.6in} | >{\centering\arraybackslash}m{3in} | >{\centering\arraybackslash}m{0.1in} | >{\centering\arraybackslash}m{0.1in} |}
\hline
\rowcolor{RoyalBlue} 
\textbf{ID} & \textbf{Título de la Historia} & \textbf{E} & \textbf{P} \\ \hline
H.U. 8.1  & \multicolumn{1}{p{3in}|}{El usuario debe poder visualizar la popularidad que está teniendo la tendencia por medio de una gráfica de área.} & 8  & 1  \\ \hline
H.U. 9.1  & \multicolumn{1}{p{3in}|}{El usuario verá los datos representados mediante un gráfico de radial de barras.} & 8  & 1  \\ \hline
H.U. 11.1  & \multicolumn{1}{p{3in}|}{El usuario debe poder ver los sentimientos generales, a partir de los tweets recogidos referentes a la tendencia, mediante una gráfica de burbujas.} & 9  & 2  \\ \hline
\end{tabular}
\caption[Subniveles de Títulos de Historias de Usuario]{Subniveles de Títulos de Historias de Usuario.}
\end{table}

\newpage

\section{Sprint Backlog}
Quedando los \textit{Product Backlog} concluidos, ahora podemos dividirlos en los distintos \textit{Sprint} acordados previamente.

\subsection{Sprint 1}
\begin{table}[H]
\centering
\small
\begin{tabular}{| >{\centering\arraybackslash}m{0.55in} | >{\centering\arraybackslash}m{3in} | >{\centering\arraybackslash}m{0.1in} | >{\centering\arraybackslash}m{0.1in} |}
\hline
\rowcolor{RoyalBlue} 
\textbf{ID} & \textbf{Título de la Historia} & \textbf{E} & \textbf{P} \\ \hline
H.U. 1  & \multicolumn{1}{p{3in}|}{Cualquier usuario puede usar la aplicación sin necesidad de registrarse.}   & 1 & 1  \\ \hline
H.U. 6  & \multicolumn{1}{p{3in}|}{El usuario debe saber, por medio de alertas, si el servicio o los parámetros no han funcionado.} & 3  & 3  \\ \hline
\end{tabular}
\caption[Títulos de Sprint 1]{Títulos del Sprint 1.}
\end{table}

\begin{table}[H]
\centering
\small
\begin{tabular}{| >{\centering\arraybackslash}m{0.9in} >{\centering\arraybackslash}m{0.9in} >{\centering\arraybackslash}m{0.9in} >{\centering\arraybackslash}m{0.9in} |}
\hline
\multicolumn{1}{|p{0.9in}|}{\cellcolor{RoyalBlue}\textbf{H.U. 1}} & \multicolumn{3}{p{2.7in}|}{Cualquier usuario puede usar la aplicación sin
necesidad de registrarse.} \\ \hline
\multicolumn{4}{|p{3.8in}|}{\textbf{Descripción}: El usuario debe poder entrar a la página web y poder visualizar su contenido sin ningún tipo de restricción. Esto debido a que el contenido aportado por la aplicación web debe centrarse al público general.} \\ \hline
\multicolumn{2}{|p{1.8in}|}{\textbf{Estimación}: 1} & \multicolumn{2}{p{1.8in}|}{\textbf{Prioridad}: 1} \\ \hline
\multicolumn{4}{|p{3.6in}|}{\textbf{Pruebas de aceptación}:
    \begin{itemize}
        \item Entrar a cualquier URL dinámico de la página y poder obtener una respuesta.
        \item Ante un error de carga de contenido o una URL mal escrita, el usuario deberá saber el error específico.
    \end{itemize}
} \\ \hline
\multicolumn{4}{|p{3.6in}|}{\textbf{Tareas}:
    \begin{itemize}
        \item Instalar Vue.js, sus dependencias y el enrutador.
        \item Instalar un lint adecuado para Vue.
        \item Crear rutas y composiciones que devuelvan una página con contenido.
    \end{itemize}
} \\ \hline
\multicolumn{4}{|p{3.6in}|}{\textbf{Observaciones}:
    \begin{itemize}
        \item Tener en cuenta rutas y errores por defecto.
    \end{itemize}
} \\ \hline
\end{tabular}
\caption[Descomposición de la Historia de Usuario 1]{Descomposición de la Historia de Usuario 1.}
\end{table}

\begin{table}[H]
\centering
\small
\begin{tabular}{| >{\centering\arraybackslash}m{0.9in} >{\centering\arraybackslash}m{0.9in} >{\centering\arraybackslash}m{0.9in} >{\centering\arraybackslash}m{0.9in} |}
\hline
\multicolumn{1}{|p{0.9in}|}{\cellcolor{RoyalBlue}\textbf{H.U. 6}} & \multicolumn{3}{p{2.7in}|}{El usuario debe saber, por medio de alertas, si el
servicio o los parámetros no han funcionado.} \\ \hline
\multicolumn{4}{|p{3.8in}|}{\textbf{Descripción}: El usuario debe poder entrar a la página web y poder si está funcionando o no. En caso de fallo, deberá poder ver un mensaje de error propiamente explicado el fallo en concreto.} \\ \hline
\multicolumn{2}{|p{1.8in}|}{\textbf{Estimación}: 3} & \multicolumn{2}{p{1.8in}|}{\textbf{Prioridad}: 3} \\ \hline
\multicolumn{4}{|p{3.6in}|}{\textbf{Pruebas de aceptación}:
    \begin{itemize}
        \item Poder entrar a diferentes rutas por erróneas o cuando el servicio esté caído y diferenciar diferentes errores.
        \item Diferenciar errores como la carga de contenido, el propio funcionamiento del servidor o contenido erróneo.
    \end{itemize}
} \\ \hline
\multicolumn{4}{|p{3.6in}|}{\textbf{Tareas}:
    \begin{itemize}
        \item Crear rutas por defecto, con alertas definidas para diferentes tipos de errores.
    \end{itemize}
} \\ \hline
\multicolumn{4}{|p{3.6in}|}{\textbf{Observaciones}:
    \begin{itemize}
        \item Tener en cuenta todos los parámetros a la hora de realizar peticiones al servidor, como el parámetro de \textit{status} de una petición.
    \end{itemize}
} \\ \hline
\end{tabular}
\caption[Descomposición de la Historia de Usuario 6]{Descomposición de la Historia de Usuario 6.}
\end{table}

\newpage

\subsection{Sprint 2}\label{subs:sprint-2}
\begin{table}[H]
\centering
\small
\begin{tabular}{| >{\centering\arraybackslash}m{0.55in} | >{\centering\arraybackslash}m{3in} | >{\centering\arraybackslash}m{0.1in} | >{\centering\arraybackslash}m{0.1in} |}
\hline
\rowcolor{RoyalBlue} 
\textbf{ID} & \textbf{Título de la Historia} & \textbf{E} & \textbf{P} \\ \hline
H.U. 2  & \multicolumn{1}{p{3in}|}{El usuario debe poder visualizar las diez tendencias más populares por defecto.}   & 5 & 1  \\ \hline
\end{tabular}
\caption[Títulos de Sprint 2]{Títulos del Sprint 2.}
\end{table}

\begin{table}[H]
\centering
\small
\begin{tabular}{| >{\centering\arraybackslash}m{0.9in} >{\centering\arraybackslash}m{0.9in} >{\centering\arraybackslash}m{0.9in} >{\centering\arraybackslash}m{0.9in} |}
\hline
\multicolumn{1}{|p{0.9in}|}{\cellcolor{RoyalBlue}\textbf{H.U. 2}} & \multicolumn{3}{p{2.7in}|}{El usuario debe poder visualizar las diez tendencias más populares por defecto.} \\ \hline
\multicolumn{4}{|p{3.8in}|}{\textbf{Descripción}: El usuario debe poder visualizar el contenido de tendencias de un país en el día actual en la página de defecto, sin tener que poner ningún parámetro.} \\ \hline
\multicolumn{2}{|p{1.8in}|}{\textbf{Estimación}: 5} & \multicolumn{2}{p{1.8in}|}{\textbf{Prioridad}: 1} \\ \hline
\multicolumn{4}{|p{3.6in}|}{\textbf{Pruebas de aceptación}:
    \begin{itemize}
        \item Entrar a la página web por defecto que haya una petición correcta al servidor.
        \item Que el servidor devuelva las tendencias de un país y fecha del día actual.
    \end{itemize}
} \\ \hline
\multicolumn{4}{|p{3.6in}|}{\textbf{Tareas}:
    \begin{itemize}
        \item Instalar Tweepy, sus dependencias y aprender las diferentes consultas a la API de Twitter.
        \item Filtrar las tendencias parecidas o iguales. Diferenciando tildes, minúsculas, mayúsculas o simplemente frases más largas.
        \item Manejar las diferentes entidades a la hora de construir el contenido a devolver para el usuario.
        \item Manejar una Base de Datos Mongo y definir distintos módulos para su correcto funcionamiento.
    \end{itemize}
} \\ \hline
\multicolumn{4}{|p{3.6in}|}{\textbf{Observaciones}:
    \begin{itemize}
        \item La Base de Datos se tendrá que actualizar cada hora.
        \item Al momento de actualizar la Base de Datos se debe realizar una Unión y Diferencia Simétrica de los datos viejo y nuevos. Con la Unión se actualizan los datos viejos y con la Diferencia Simétrica, agregamos los datos nuevos junto a los viejos.
    \end{itemize}
} \\ \hline
\end{tabular}
\caption[Descomposición de la Historia de Usuario 2]{Descomposición de la Historia de Usuario 2.}
\end{table}

\newpage

\subsection{Sprint 3}
\begin{table}[ht]
\centering
\small
\begin{tabular}{| >{\centering\arraybackslash}m{0.55in} | >{\centering\arraybackslash}m{3in} | >{\centering\arraybackslash}m{0.1in} | >{\centering\arraybackslash}m{0.1in} |}
\hline
\rowcolor{RoyalBlue} 
\textbf{ID} & \textbf{Título de la Historia} & \textbf{E} & \textbf{P} \\ \hline
H.U. 3  & \multicolumn{1}{p{3in}|}{El usuario debe poder visualizar las diez tendencias más populares, pudiendo seleccionar el país como parámetro.} & 5  & 1  \\ \hline
H.U. 4  & \multicolumn{1}{p{3in}|}{El usuario debe poder visualizar las diez tendencias más populares, pudiendo seleccionar la fecha como parámetro.} & 5  & 1  \\ \hline
\end{tabular}
\caption[Títulos de Sprint 3]{Títulos del Sprint 3.}
\end{table}

\begin{table}[H]
\centering
\small
\begin{tabular}{| >{\centering\arraybackslash}m{0.9in} >{\centering\arraybackslash}m{0.9in} >{\centering\arraybackslash}m{0.9in} >{\centering\arraybackslash}m{0.9in} |}
\hline
\multicolumn{1}{|p{0.9in}|}{\cellcolor{RoyalBlue}\textbf{H.U. 3}} & \multicolumn{3}{p{2.7in}|}{El usuario debe poder visualizar las diez tendencias más populares, pudiendo seleccionar el país como parámetro.} \\ \hline
\multicolumn{4}{|p{3.8in}|}{\textbf{Descripción}: El usuario debe poder entrar a la página web y poder visualizar el contenido, aportando el parámetro del país donde quiera buscar la tendencia.} \\ \hline
\multicolumn{2}{|p{1.8in}|}{\textbf{Estimación}: 5} & \multicolumn{2}{p{1.8in}|}{\textbf{Prioridad}: 1} \\ \hline
\multicolumn{4}{|p{3.6in}|}{\textbf{Pruebas de aceptación}:
    \begin{itemize}
        \item Entrar al URL dinámico de la página como país como parámetro y que el servidor devuelva el objeto correspondiente.
        \item Ante un error de carga de contenido o una URL mal escrita, el usuario deberá saber el error específico.
    \end{itemize}
} \\ \hline
\multicolumn{4}{|p{3.6in}|}{\textbf{Tareas}:
    \begin{itemize}
        \item Configuración de respectivos enrutadores. Tanto en la parte Front-end y Back-end.
        \item Se deben poder buscar en la Base de Datos por países, con su fecha correspondiente. De modo que ya debe haber objetos creados con país como valor identificativo.
    \end{itemize}
} \\ \hline
\multicolumn{4}{|p{3.6in}|}{\textbf{Observaciones}:
    \begin{itemize}
    \item
    Ninguna.
    \end{itemize}
} \\ \hline
\end{tabular}
\caption[Descomposición de la Historia de Usuario 3]{Descomposición de la Historia de Usuario 3.}
\end{table}

\begin{table}[H]
\centering
\small
\begin{tabular}{| >{\centering\arraybackslash}m{0.9in} >{\centering\arraybackslash}m{0.9in} >{\centering\arraybackslash}m{0.9in} >{\centering\arraybackslash}m{0.9in} |}
\hline
\multicolumn{1}{|p{0.9in}|}{\cellcolor{RoyalBlue}\textbf{H.U. 4}} & \multicolumn{3}{p{2.7in}|}{El usuario debe poder visualizar las diez tendencias más populares, pudiendo seleccionar el país como parámetro.} \\ \hline
\multicolumn{4}{|p{3.8in}|}{\textbf{Descripción}: El usuario debe poder visualizar las diez tendencias más populares, pudiendo seleccionar la fecha como parámetro.} \\ \hline
\multicolumn{2}{|p{1.8in}|}{\textbf{Estimación}: 5} & \multicolumn{2}{p{1.8in}|}{\textbf{Prioridad}: 1} \\ \hline
\multicolumn{4}{|p{3.6in}|}{\textbf{Pruebas de aceptación}:
    \begin{itemize}
        \item Entrar al URL dinámico de la página como fecha como parámetro y que el servidor devuelva el objeto correspondiente.
        \item Ante un error de carga de contenido o una URL mal escrita, el usuario deberá saber el error específico.
    \end{itemize}
} \\ \hline
\multicolumn{4}{|p{3.6in}|}{\textbf{Tareas}:
    \begin{itemize}
        \item Configuración de respectivos enrutadores. Tanto en la parte Front-end y Back-end.
        \item Se deben poder buscar en la Base de Datos por fecha, con su país correspondiente. De modo que ya debe haber objetos creados con fecha como valor identificativo.
    \end{itemize}
} \\ \hline
\multicolumn{4}{|p{3.6in}|}{\textbf{Observaciones}:
    \begin{itemize}
    \item
    Ninguna.
    \end{itemize}
} \\ \hline
\end{tabular}
\caption[Descomposición de la Historia de Usuario 4]{Descomposición de la Historia de Usuario 4.}
\end{table}

\newpage

\subsection{Sprint 4}
\begin{table}[H]
\centering
\small
\begin{tabular}{| >{\centering\arraybackslash}m{0.55in} | >{\centering\arraybackslash}m{3in} | >{\centering\arraybackslash}m{0.1in} | >{\centering\arraybackslash}m{0.1in} |}
\hline
\rowcolor{RoyalBlue} 
\textbf{ID} & \textbf{Título de la Historia} & \textbf{E} & \textbf{P} \\ \hline
H.U. 5  & \multicolumn{1}{p{3in}|}{El usuario puede buscar sus propios tópicos y visualizarlos del mismo modo que las tendencias.} & 4  & 2  \\ \hline
\end{tabular}
\caption[Títulos de Sprint 4]{Títulos del Sprint 4.}
\end{table}

\begin{table}[H]
\centering
\small
\begin{tabular}{| >{\centering\arraybackslash}m{0.9in} >{\centering\arraybackslash}m{0.9in} >{\centering\arraybackslash}m{0.9in} >{\centering\arraybackslash}m{0.9in} |}
\hline
\multicolumn{1}{|p{0.9in}|}{\cellcolor{RoyalBlue}\textbf{H.U. 5}} & \multicolumn{3}{p{2.7in}|}{El usuario puede buscar sus propios tópicos y visualizarlos del mismo modo que las tendencias.} \\ \hline
\multicolumn{4}{|p{3.8in}|}{\textbf{Descripción}: El usuario puede buscar sus propios tópicos, mediante un buscador general, y visualizarlos del mismo modo que las tendencias. Con las mismas características que las tendencias, menos la popularidad, que solo se puede calcular si es una tendencia, un dato proveniente de Twitter.} \\ \hline
\multicolumn{2}{|p{1.8in}|}{\textbf{Estimación}: 4} & \multicolumn{2}{p{1.8in}|}{\textbf{Prioridad}: 2} \\ \hline
\multicolumn{4}{|p{3.6in}|}{\textbf{Pruebas de aceptación}:
    \begin{itemize}
        \item Que el usuario identifique el buscador y pueda escribir sus propios tópicos que desee buscar.
        \item Que el usuario pueda cambiar de país como parámetro al buscador para los tópicos.
    \end{itemize}
} \\ \hline
\multicolumn{4}{|p{3.6in}|}{\textbf{Tareas}:
    \begin{itemize}
        \item Adaptar el diseño para el buscador en Front-end.
        \item Adaptar la ruta correspondiente en forma de \textit{query} y que el servidor sea capaz de procesarla.
    \end{itemize}
} \\ \hline
\multicolumn{4}{|p{3.6in}|}{\textbf{Observaciones}:
    \begin{itemize}
    \item El tópico no se guarda en la Base de Datos, sino que se devuelve cada vez que se inicia una búsqueda.
    \end{itemize}
} \\ \hline
\end{tabular}
\caption[Descomposición de la Historia de Usuario 5]{Descomposición de la Historia de Usuario 5.}
\end{table}

\newpage

\subsection{Sprint 5}
\begin{table}[H]
\centering
\small
\begin{tabular}{| >{\centering\arraybackslash}m{0.55in} | >{\centering\arraybackslash}m{3in} | >{\centering\arraybackslash}m{0.1in} | >{\centering\arraybackslash}m{0.1in} |}
\hline
\rowcolor{RoyalBlue} 
\textbf{ID} & \textbf{Título de la Historia} & \textbf{E} & \textbf{P} \\ \hline
H.U. 7  & \multicolumn{1}{p{3in}|}{El usuario tiene que poder navegar intuitivamente, es decir por medio de gestos de \textit{scroll}, por la página.} & 7  & 1  \\ \hline
\end{tabular}
\caption[Títulos de Sprint 5]{Títulos del Sprint 5.}
\end{table}

\begin{table}[H]
\centering
\small
\begin{tabular}{| >{\centering\arraybackslash}m{0.9in} >{\centering\arraybackslash}m{0.9in} >{\centering\arraybackslash}m{0.9in} >{\centering\arraybackslash}m{0.9in} |}
\hline
\multicolumn{1}{|p{0.9in}|}{\cellcolor{RoyalBlue}\textbf{H.U. 7}} & \multicolumn{3}{p{2.7in}|}{El usuario tiene que poder navegar intuitivamente, es decir por medio de gestos de \textit{scroll}, por la página.} \\ \hline
\multicolumn{4}{|p{3.8in}|}{\textbf{Descripción}: El usuario tiene que poder navegar intuitivamente. Tanto haciendo \textit{scroll} verticalmente o horizontalmente por la página web. Además, el contenido debe desplazarse automáticamente al centro de la pantalla, ocupando el mayor tamaño de espacio disponible. Todo esto por medio de gestos, tanto en una pantalla grande o en un móvil.} \\ \hline
\multicolumn{2}{|p{1.8in}|}{\textbf{Estimación}: 7} & \multicolumn{2}{p{1.8in}|}{\textbf{Prioridad}: 1} \\ \hline
\multicolumn{4}{|p{3.6in}|}{\textbf{Pruebas de aceptación}:
    \begin{itemize}
        \item Que el usuario pueda hacer \textit{scroll} verticalmente y el contenido se mueva verticalmente dependiendo del gesto del usuario.
        \item Que el usuario pueda hacer \textit{scroll} horizontalmente y el contenido se mueva horizontalmente dependiendo del gesto del usuario.
    \end{itemize}
} \\ \hline
\multicolumn{4}{|p{3.6in}|}{\textbf{Tareas}:
    \begin{itemize}
        \item Adaptar el diseño en Front-end. Esto se hará gracias al \textit{framework} Tailwind CSS. Consiguiendo que el gesto del usuario sea un \textit{scroll} automático y el contenido se centre en medio de la pantalla siempre después de procesar cualquier gesto.
        \item Adaptar Vue para que la página web pueda proporcionar contenido dinámico, es decir, un vector de tendencias que puede tener diferente tamaño cada vez que se haga una petición al servidor.
    \end{itemize}
} \\ \hline
\multicolumn{4}{|p{3.6in}|}{\textbf{Observaciones}:
    \begin{itemize}
    \item El diseño, al ser de cartas o módulos es fácilmente adaptable, pero hay que tener en cuenta las diferentes versiones de escritorio o móvil.
    \end{itemize}
} \\ \hline
\end{tabular}
\caption[Descomposición de la Historia de Usuario 7]{Descomposición de la Historia de Usuario 7.}
\end{table}

\newpage

\subsection{Sprint 6}\label{subs:sprint-6}
\begin{table}[H]
\centering
\small
\begin{tabular}{| >{\centering\arraybackslash}m{0.55in} | >{\centering\arraybackslash}m{3in} | >{\centering\arraybackslash}m{0.1in} | >{\centering\arraybackslash}m{0.1in} |}
\hline
\rowcolor{RoyalBlue} 
\textbf{ID} & \textbf{Título de la Historia} & \textbf{E} & \textbf{P} \\ \hline
H.U. 8  & \multicolumn{1}{p{3in}|}{El usuario debe poder analizar datos de interés sobre la popularidad.} & 8  & 1  \\ \hline
H.U. 8.1  & \multicolumn{1}{p{3in}|}{El usuario debe poder visualizar la popularidad que está teniendo la tendencia por medio de una gráfica de área.} & 8  & 1  \\ \hline
\end{tabular}
\caption[Títulos de Sprint 6]{Títulos del Sprint 6.}
\end{table}

\begin{table}[H]
\centering
\small
\begin{tabular}{| >{\centering\arraybackslash}m{0.9in} >{\centering\arraybackslash}m{0.9in} >{\centering\arraybackslash}m{0.9in} >{\centering\arraybackslash}m{0.9in} |}
\hline
\multicolumn{1}{|p{0.9in}|}{\cellcolor{RoyalBlue}\textbf{H.U. 8}} & \multicolumn{3}{p{2.7in}|}{El usuario debe poder analizar datos de interés sobre la popularidad.} \\ \hline
\multicolumn{4}{|p{3.8in}|}{\textbf{Descripción}: Uno de los contenidos principales sobre una tendencia para el usuario serán datos de interés sobre la popularidad de la tendencia. Datos como la moda, la media y el primer o último dato calculado.} \\ \hline
\multicolumn{2}{|p{1.8in}|}{\textbf{Estimación}: 8} & \multicolumn{2}{p{1.8in}|}{\textbf{Prioridad}: 1} \\ \hline
\multicolumn{4}{|p{3.6in}|}{\textbf{Pruebas de aceptación}:
    \begin{itemize}
        \item Que el usuario pueda informarse debidamente de la media o la popularidad promedio de la tendencia.
        \item Que el usuario pueda informarse debidamente de la moda o a popularidad más grande alcanzada de la tendencia.
        \item Que el usuario pueda informarse debidamente del primer y último volumen de popularidad calculados de la tendencia.
    \end{itemize}
} \\ \hline
\multicolumn{4}{|p{3.6in}|}{\textbf{Tareas}:
    \begin{itemize}
        \item Adaptar el diseño en Front-end al diseño de la carta y desarrollar funciones para que el número tenga un separador de millares para una lectura más fácil.
        \item Crear el modelo correspondiente en la Base de Datos.
    \end{itemize}
} \\ \hline
\multicolumn{4}{|p{3.6in}|}{\textbf{Observaciones}:
    \begin{itemize}
    \item Ninguna.
    \end{itemize}
} \\ \hline
\end{tabular}
\caption[Descomposición de la Historia de Usuario 8]{Descomposición de la Historia de Usuario 8.}
\end{table}


\begin{table}[H]
\centering
\small
\begin{tabular}{| >{\centering\arraybackslash}m{0.9in} >{\centering\arraybackslash}m{0.9in} >{\centering\arraybackslash}m{0.9in} >{\centering\arraybackslash}m{0.9in} |}
\hline
\multicolumn{1}{|p{0.9in}|}{\cellcolor{RoyalBlue}\textbf{H.U. 8.1}} & \multicolumn{3}{p{2.7in}|}{El usuario debe poder visualizar la popularidad que está teniendo la tendencia por medio de una gráfica de área.} \\ \hline
\multicolumn{4}{|p{3.8in}|}{\textbf{Descripción}: Uno de los contenidos principales sobre una tendencia para el usuario será la gráfica de área que representará la propia popularidad de la tendencia. El eje X será la hora a la que se ha calculado la popularidad y el eje Y será la popularidad de la tendencia en cuestión.} \\ \hline
\multicolumn{2}{|p{1.8in}|}{\textbf{Estimación}: 8} & \multicolumn{2}{p{1.8in}|}{\textbf{Prioridad}: 1} \\ \hline
\multicolumn{4}{|p{3.6in}|}{\textbf{Pruebas de aceptación}:
    \begin{itemize}
        \item Que el usuario pueda informarse debidamente con el contenido gráfico sobre la popularidad de una tendencia en un momento dado.
    \end{itemize}
} \\ \hline
\multicolumn{4}{|p{3.6in}|}{\textbf{Tareas}:
    \begin{itemize}
        \item Adaptar el diseño en Front-end. Gracias a la librería ApexCharts.js, podremos mostrar al usuario un gráfico solamente necesitando dos listas, el eje X y el eje Y que habrán sido previamente calculados.
        \item Se adaptará el gráfico al diseño de la carta, además se apagará toda la interactividad con él porque su función solo es gráfica.
    \end{itemize}
} \\ \hline
\multicolumn{4}{|p{3.6in}|}{\textbf{Observaciones}:
    \begin{itemize}
    \item El diseño tiene que ser \textit{responsive}, afectando también al gráfico.
    \end{itemize}
} \\ \hline
\end{tabular}
\caption[Descomposición de la Historia de Usuario 8.1]{Descomposición de la Historia de Usuario 8.1.}
\end{table}

\newpage

\subsection{Sprint 7}\label{subs:sprint-7}
\begin{table}[H]
\centering
\small
\begin{tabular}{| >{\centering\arraybackslash}m{0.55in} | >{\centering\arraybackslash}m{3in} | >{\centering\arraybackslash}m{0.1in} | >{\centering\arraybackslash}m{0.1in} |}
\hline
\rowcolor{RoyalBlue} 
\textbf{ID} & \textbf{Título de la Historia} & \textbf{E} & \textbf{P} \\ \hline
H.U. 9  & \multicolumn{1}{p{3in}|}{El usuario debe poder visualizar las palabras más repetidas o comunes mediante un porcentaje, a partir de los tweets recogidos referentes a la tendencia.} & 8  & 1  \\ \hline
H.U. 9.1  & \multicolumn{1}{p{3in}|}{El usuario verá los datos representados mediante un gráfico de radial de barras.} & 8  & 1  \\ \hline
H.U. 10  & \multicolumn{1}{p{3in}|}{El usuario debe poder visualizar las \textit{keywords}, a partir de los tweets recogidos referentes a la tendencia.} & 8  & 1  \\ \hline
\end{tabular}
\caption[Títulos de Sprint 7]{Títulos del Sprint 7.}
\end{table}

\begin{table}[H]
\centering
\small
\begin{tabular}{| >{\centering\arraybackslash}m{0.9in} >{\centering\arraybackslash}m{0.9in} >{\centering\arraybackslash}m{0.9in} >{\centering\arraybackslash}m{0.9in} |}
\hline
\multicolumn{1}{|p{0.9in}|}{\cellcolor{RoyalBlue}\textbf{H.U. 9}} & \multicolumn{3}{p{2.7in}|}{El usuario debe poder visualizar las palabras más repetidas o comunes, a partir de los tweets recogidos referentes a la tendencia.} \\ \hline
\multicolumn{4}{|p{3.8in}|}{\textbf{Descripción}: Uno de los contenidos principales sobre una tendencia para el usuario será visualizar las palabras más repetidas o comunes de una tendencia. Mostrado en porcentaje de aparición respecto a todas las publicaciones o tweets.} \\ \hline
\multicolumn{2}{|p{1.8in}|}{\textbf{Estimación}: 8} & \multicolumn{2}{p{1.8in}|}{\textbf{Prioridad}: 1} \\ \hline
\multicolumn{4}{|p{3.6in}|}{\textbf{Pruebas de aceptación}:
    \begin{itemize}
        \item El usuario debe informarse debidamente de las palabras más repetidas a través de cálculos de porcentaje sobre las apariciones de las palabras en los tweets.
    \end{itemize}
} \\ \hline
\multicolumn{4}{|p{3.6in}|}{\textbf{Tareas}:
    \begin{itemize}
        \item Adaptar el diseño en Front-end y que conjunte con los demás elementos.
        \item Filtrar las palabras significativas, eliminando las que pertenezcan al grupo de las \textit{stop words} o palabras vacías, es decir las que no suelen aportar significado a la hora de analizar textos. Posteriormente sólo se enseñarán las seis palabras más comunes.
        \item Crear el modelo correspondiente en la Base de Datos.
    \end{itemize}
} \\ \hline
\multicolumn{4}{|p{3.6in}|}{\textbf{Observaciones}:
    \begin{itemize}
    \item El diseño tiene que ser \textit{responsive}.
    \end{itemize}
} \\ \hline
\end{tabular}
\caption[Descomposición de la Historia de Usuario 9]{Descomposición de la Historia de Usuario 9.}
\end{table}

\begin{table}[H]
\centering
\small
\begin{tabular}{| >{\centering\arraybackslash}m{0.9in} >{\centering\arraybackslash}m{0.9in} >{\centering\arraybackslash}m{0.9in} >{\centering\arraybackslash}m{0.9in} |}
\hline
\multicolumn{1}{|p{0.9in}|}{\cellcolor{RoyalBlue}\textbf{H.U. 9.1}} & \multicolumn{3}{p{2.7in}|}{El usuario verá los datos representados mediante un gráfico de radial de barras.} \\ \hline
\multicolumn{4}{|p{3.8in}|}{\textbf{Descripción}: Otro contenido único, que aportará más significado a la hora de visualizar las palabras más repetidas o comunes de una tendencia será un gráfico de radial de barras, donde cada barra representará la palabra y el porcentaje de sus repeticiones.} \\ \hline
\multicolumn{2}{|p{1.8in}|}{\textbf{Estimación}: 8} & \multicolumn{2}{p{1.8in}|}{\textbf{Prioridad}: 1} \\ \hline
\multicolumn{4}{|p{3.6in}|}{\textbf{Pruebas de aceptación}:
    \begin{itemize}
        \item El usuario debe informarse debidamente de las palabras más repetidas a través de un gráfico de radial de barras.
        \item El usuario debe poder comparar todas las palabras repetidas. También saber cuál es la que tiene más valor y la que menos.
    \end{itemize}
} \\ \hline
\multicolumn{4}{|p{3.6in}|}{\textbf{Tareas}:
    \begin{itemize}
        \item Adaptar el diseño en Front-end, gracias a la librería ApexCharts.js. 
        \item Se debe configurar el gráfico de tal manera, que cada palabra y su valor tengan una representación única.
    \end{itemize}
} \\ \hline
\multicolumn{4}{|p{3.6in}|}{\textbf{Observaciones}:
    \begin{itemize}
    \item El diseño tiene que ser \textit{responsive}, afectando también al gráfico.
    \end{itemize}
} \\ \hline
\end{tabular}
\caption[Descomposición de la Historia de Usuario 9.1]{Descomposición de la Historia de Usuario 9.1.}
\end{table}

\begin{table}[H]
\centering
\small
\begin{tabular}{| >{\centering\arraybackslash}m{0.9in} >{\centering\arraybackslash}m{0.9in} >{\centering\arraybackslash}m{0.9in} >{\centering\arraybackslash}m{0.9in} |}
\hline
\multicolumn{1}{|p{0.9in}|}{\cellcolor{RoyalBlue}\textbf{H.U. 10}} & \multicolumn{3}{p{2.7in}|}{El usuario debe poder visualizar las \textit{keywords}, a partir de los tweets recogidos referentes a la tendencia.} \\ \hline
\multicolumn{4}{|p{3.8in}|}{\textbf{Descripción}: Uno de los contenidos principales sobre una tendencia para el usuario será visualizar las \textit{keywords} más relevantes sobre los textos recogidos de las publicaciones o tweets.} \\ \hline
\multicolumn{2}{|p{1.8in}|}{\textbf{Estimación}: 8} & \multicolumn{2}{p{1.8in}|}{\textbf{Prioridad}: 1} \\ \hline
\multicolumn{4}{|p{3.6in}|}{\textbf{Pruebas de aceptación}:
    \begin{itemize}
        \item El usuario debe informarse de cuáles son las \textit{keywords} más relevantes de una tendencia.
        \item Las \textit{keywords} deben expresar al usuario una perspectiva de lo que es relevante o de importancia en el conjunto de las publicaciones o tweets de la tendencia.
    \end{itemize}
} \\ \hline
\multicolumn{4}{|p{3.6in}|}{\textbf{Tareas}:
    \begin{itemize}
        \item Adaptar el diseño en Front-end.
        \item Configurar el algoritmo \ac{YAKE}, adaptándolo a extraer un conjunto de textos que serán parecidos, ya que son publicaciones sobre un mismo tópico.
        \item Filtrar el texto antes de extraer las \textit{keywords} con el algoritmo \ac{YAKE}.
        \item Filtrar las \textit{keywords} parecidas o iguales, luego enseñar solo las tres más relevantes.
    \end{itemize}
} \\ \hline
\multicolumn{4}{|p{3.6in}|}{\textbf{Observaciones}:
    \begin{itemize}
    \item El diseño tiene que ser \textit{responsive}.
    \end{itemize}
} \\ \hline
\end{tabular}
\caption[Descomposición de la Historia de Usuario 10]{Descomposición de la Historia de Usuario 10.}
\end{table}

\newpage

\subsection{Sprint 8}\label{subs:sprint-8}
\begin{table}[H]
\centering
\small
\begin{tabular}{| >{\centering\arraybackslash}m{0.6in} | >{\centering\arraybackslash}m{3in} | >{\centering\arraybackslash}m{0.1in} | >{\centering\arraybackslash}m{0.1in} |}
\hline
\rowcolor{RoyalBlue} 
\textbf{ID} & \textbf{Título de la Historia} & \textbf{E} & \textbf{P} \\ \hline
H.U. 11  & \multicolumn{1}{p{3in}|}{El usuario puede analizar los sentimientos generales mediante un porcentaje mostrado.} & 9  & 2  \\ \hline
H.U. 11.1  & \multicolumn{1}{p{3in}|}{El usuario debe poder ver los sentimientos generales, a partir de los tweets recogidos referentes a la tendencia, mediante una gráfica de burbujas.} & 9  & 2  \\ \hline
\end{tabular}
\caption[Títulos de Sprint 8]{Títulos del Sprint 8.}
\end{table}

\begin{table}[H]
\centering
\small
\begin{tabular}{| >{\centering\arraybackslash}m{0.9in} >{\centering\arraybackslash}m{0.9in} >{\centering\arraybackslash}m{0.9in} >{\centering\arraybackslash}m{0.9in} |}
\hline
\multicolumn{1}{|p{0.9in}|}{\cellcolor{RoyalBlue}\textbf{H.U. 11}} & \multicolumn{3}{p{2.7in}|}{El usuario puede analizar los sentimientos generales mediante un porcentaje mostrado.} \\ \hline
\multicolumn{4}{|p{3.8in}|}{\textbf{Descripción}: Uno de los contenidos principales sobre una tendencia para el usuario será visualizar el sentimiento general que tienen los textos recogidos de las publicaciones o tweets. En este caso, será mostrado mediante un porcentaje.} \\ \hline
\multicolumn{2}{|p{1.8in}|}{\textbf{Estimación}: 9} & \multicolumn{2}{p{1.8in}|}{\textbf{Prioridad}: 2} \\ \hline
\multicolumn{4}{|p{3.6in}|}{\textbf{Pruebas de aceptación}:
    \begin{itemize}
        \item El usuario debe informarse de cuál es el sentimiento general de una tendencia y saber cuál es el porcentaje correspondiente de los sentimientos positivos, negativos y neutrales.
    \end{itemize}
} \\ \hline
\multicolumn{4}{|p{3.6in}|}{\textbf{Tareas}:
    \begin{itemize}
        \item Mostrar al usuario tres tipos de porcentaje (positivos, negativos y neutrales). El porcentaje será calculado mediante el tamaño de la burbuja, ya que el tamaño o radio representa su relevancia.
        \item Crear el modelo correspondiente en la Base de Datos.
    \end{itemize}
} \\ \hline
\multicolumn{4}{|p{3.6in}|}{\textbf{Observaciones}:
    \begin{itemize}
    \item El diseño tiene que ser \textit{responsive} y adaptarse al contenido gráfico con sus colores correspondientes.
    \end{itemize}
} \\ \hline
\end{tabular}
\caption[Descomposición de la Historia de Usuario 11]{Descomposición de la Historia de Usuario 11.}
\end{table}

\begin{table}[H]\label{subs:HU-111}
\centering
\small
\begin{tabular}{| >{\centering\arraybackslash}m{0.9in} >{\centering\arraybackslash}m{0.9in} >{\centering\arraybackslash}m{0.9in} >{\centering\arraybackslash}m{0.9in} |}
\hline
\multicolumn{1}{|p{0.9in}|}{\cellcolor{RoyalBlue}\textbf{H.U. 11.1}} & \multicolumn{3}{p{2.7in}|}{El usuario debe poder ver los sentimientos generales, a partir de los tweets recogidos referentes a la tendencia, mediante una gráfica de burbujas.} \\ \hline
\multicolumn{4}{|p{3.8in}|}{\textbf{Descripción}: Uno de los contenidos principales sobre una tendencia para el usuario será visualizar el sentimiento general que tienen los textos recogidos de las publicaciones o tweets. El análisis de sentimientos será basado en reglas de léxicos y se mostrará mediante un gráfico.} \\ \hline
\multicolumn{2}{|p{1.8in}|}{\textbf{Estimación}: 9} & \multicolumn{2}{p{1.8in}|}{\textbf{Prioridad}: 2} \\ \hline
\multicolumn{4}{|p{3.6in}|}{\textbf{Pruebas de aceptación}:
    \begin{itemize}
        \item El usuario debe informarse de cual es el sentimiento general de una tendencia de una manera visual o gráfica. Se le mostrará al usuario un gráfico de burbujas.
    \end{itemize}
} \\ \hline
\multicolumn{4}{|p{3.6in}|}{\textbf{Tareas}:
    \begin{itemize}
        \item Configurar la herramienta \ac{VADER} y adaptar el léxico de cada país.
        \item Mostrar al usuario tres tipos de burbuja (Positivo, Negativo y Neutral). Cada burbuja tendrá un tamaño diferente dependiendo de su relevancia, mediante la popularidad de la publicación o si hay muchos sentimientos parecidos.
        \item Mostrar un posicionamiento de la burbuja relevante. El eje X será el sentimiento general de una publicación. Mientras el eje Y representará la relación con la clasificación de palabras relevantes, que fueron calculadas anteriormente.
    \end{itemize}
} \\ \hline
\multicolumn{4}{|p{3.6in}|}{\textbf{Observaciones}:
    \begin{itemize}
    \item El diseño tiene que ser \textit{responsive}.
    \end{itemize}
} \\ \hline
\end{tabular}
\caption[Descomposición de la Historia de Usuario 11.1]{Descomposición de la Historia de Usuario 11.1.}
\end{table}

\newpage

\subsection{Sprint 9}\label{subs:sprint-9}
\begin{table}[H]
\centering
\small
\begin{tabular}{| >{\centering\arraybackslash}m{0.55in} | >{\centering\arraybackslash}m{3in} | >{\centering\arraybackslash}m{0.1in} | >{\centering\arraybackslash}m{0.1in} |}
\hline
\rowcolor{RoyalBlue} 
\textbf{ID} & \textbf{Título de la Historia} & \textbf{E} & \textbf{P} \\ \hline
H.U. 12  & \multicolumn{1}{p{3in}|}{El usuario debe poder ver tres noticias más actuales relacionadas con la tendencia y poder acceder a ellas.} & 10  & 1  \\ \hline
\end{tabular}
\caption[Títulos de Sprint 9]{Títulos del Sprint 9.}
\end{table}

\begin{table}[H]
\centering
\small
\begin{tabular}{ >{\centering\arraybackslash}m{0.9in} >{\centering\arraybackslash}m{0.9in} >{\centering\arraybackslash}m{0.9in} >{\centering\arraybackslash}m{0.9in} }
\hline
\multicolumn{1}{|p{0.9in}|}{\cellcolor{RoyalBlue}\textbf{H.U. 12}} & \multicolumn{3}{p{2.7in}|}{El usuario debe poder ver tres noticias más actuales relacionadas con la tendencia y poder acceder a ellas.} \\ \hline
\multicolumn{4}{|p{3.8in}|}{\textbf{Descripción}: Uno de los contenidos principales sobre una tendencia para el usuario será visualizar las distintas noticias relevantes que pueden tener alguna relación las tendencias.} \\ \hline
\multicolumn{2}{|p{1.8in}|}{\textbf{Estimación}: 10} & \multicolumn{2}{p{1.8in}|}{\textbf{Prioridad}: 1} \\ \hline
\multicolumn{4}{|p{3.6in}|}{\textbf{Pruebas de aceptación}:
    \begin{itemize}
        \item El usuario debe informarse de tres noticias con editoriales diferentes, pero que tengan afinidad respecto a la tendencia.
        \item El usuario podrá ver una foto, un título, la editorial, la fecha de publicación y algún texto sobre el articulo o noticia.
    \end{itemize}
} \\ \hline
\multicolumn{4}{|p{3.6in}|}{\textbf{Tareas}:
    \begin{itemize}
        \item Desarrollar \textit{web scraping} o raspado web, una técnica utilizada para extraer información de sitios web. En nuestro caso se utilizará para extraer noticias relevantes desde Google News y posteriormente desde cada página del artículo.
        \item Se buscarán artículos relacionado mediante el nombre de la tendencia y algunas palabras relevantes que se extrajeron anteriormente, con el objetivo de precisar la búsqueda.
        \item Crear el modelo correspondiente en la Base de Datos.
    \end{itemize}
} \\ \hline
\multicolumn{4}{|p{3.6in}|}{\textbf{Observaciones}:
    \begin{itemize}
    \item El diseño tiene que ser \textit{responsive} y adaptarse al contenido, tanto con la foto o el texto.
    \end{itemize}
} \\ \hline
\end{tabular}
\caption[Descomposición de la Historia de Usuario 12]{Descomposición de la Historia de Usuario 12.}
\end{table}