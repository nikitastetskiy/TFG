\chapter{Contexto}\label{ch:context}
%************************************************

\section{Motivación o justificación del proyecto}

\vspace{0.3cm}

El año 1989 fue clave en distintos aspectos, conocido por la disolución del Telón de Acero en Europa y la caída del Muro de Berlín, aunque sobre todo destacó por la apertura del Internet al público. Una herramienta, en un principio, al alcance de pocos que inicio su meteórico ascenso a ser uno de los mayores inventos. Hoy en día, capaz de conectar al mundo como muy pocos y que situaciones como la reciente pandemia COVID-19 no han hecho más que impulsar el intercambio digital de información entre personas.

\vspace{0.3cm}

Para podernos situarnos con más claridad, el estudio anual de IAB Spain nos indica que en el año 2022 el 93\% de las personas que residen en España son usuarios del Internet. Además de ese porcentaje, el 85\% (28,3 millones de españoles) utilizan las redes sociales. \cite{Studio-Aloha}

\vspace{0.3cm}

Estos datos nos dejan en evidencia el hecho de que el uso de estas herramientas, tales como el Internet y las redes sociales, van en aumento. Aunque no nos debería de asombrar, ya que la generalización y el uso del Internet y las redes sociales viene de mucho antes. En el año 2012 se realizó un estudio en el que se mostraba que las redes sociales pueden llegar a crear una dependencia más grande aun que el tabaco. \cite{Tabaco-Study}

\vspace{0.3cm}

Incluso muchos programadores de redes sociales han admitido que el diseño de tales aplicaciones es adictivo de manera deliberada. Un ejemplo de esto puede ser la creación del scroll infinito, donde se brinda gran cantidad de información de una manera continua. \cite{Social-Deliberately}

\vspace{0.3cm}

La solución más sencilla, pero drástica, puede llegar a ser abandonar las redes sociales, aunque nos surge el problema de no estar tan \textit{actualizados} como otras personas a nuestro alrededor. Otra solución es usar en una menor medida las redes sociales, aunque esto nos lleva de nuevo al problema de cuándo parar si la aplicación que usamos tiene scroll o contenido infinito.

\vspace{0.3cm}

Ante tal necesidad de querer estar actualizados y a la vez querer saber cuando parar de consumir contenido, la solución que propongo es un consumo finito de contenido diario y lo más objetivo posible. Una aplicación web que aporte la información necesaria de las tendencias más destacables del día.

\section{Objetivos}

\subsection{Objetivo general}
La meta final de este proyecto es construir una aplicación web autosuficiente y bien estructurada, cuyo contenido se pueda actualizar diariamente de manera automática. Con el objetivo de que el usuario final pueda consumirlo de una manera eficaz y eficiente.

\vspace{0.3cm}

La información que se le brinda al usuario se basará en las tendencias del país seleccionado por él. A partir de las tendencias se deducirán datos, estadísticas, noticias y enlaces de interés.

\vspace{0.3cm}

Toda esta información deberá albergar un carácter sencillo y deberá ser fácil de consumir, alejándose de la posibilidad de ser abrumante o atosigadora.

\subsection{Objetivos específicos}

\begin{itemize}
\item El contenido que se le brinda al usuario debe de ser corto y finito, pero a la vez suficiente con el fin de informarse de manera adecuada.
\item La tendencia que se aporta al usuario proviene de la API de Twitter y tiene que ser la más destacable de la red social.
\item La aplicación web debe de ser intuitiva y fácil de usar, el diseño debe ajustarse para el formato móvil.
\item La estructura del proyecto deberá seguir el modelo API REST, siguiendo una arquitectura web por capas.
\item Cada contenido proporcionado para el usuario se brindará en un diseño de carta y ocupará el mayor espacio de pantalla posible. De esta manera se aprovecharán al máximo la información disponible, siendo esta una noticia, un dato de una estadística o un enlace de interés.
\end{itemize}

\newpage



\section{Sección en proceso !!!}
En esta sección se planificarán las tareas y los objetivos. Se crearán dos diagramas de Gantt, uno dónde se planifiquen los objetivos y uno final real. Los objetivos planteados son:
\begin{itemize}
\item Planificación y presupuesto.  DONE
\item Estado del arte:  DONE
\begin{itemize}
    \item Dominio del problema a resolver   DONE
    \item Metodologías y tecnologías de base que podrían usarse -> (Conclusión) DONE
\end{itemize}
\item Propuesta:
\begin{itemize}
    \item Metodología de desarrollo - SCRUM (profundidad) DONE
    \item Analisis, requisitos e historias de usuario (Backlogs)
    \item DevOps ? como metodologia
\end{itemize}
\item Diseño:
\begin{itemize}
    \item Explicación de la arquitectura usada, distintas capas y protocolos (diagramas y esquemas)
    \item Diagrama de paquetes, clases, bocetos ...
    \item Justificación API de Twitter (comparación con Pytrends) DONE antes
    \item Integración continua
    \begin{itemize}
        \item Gestor de paquetes: poetry
        \item Linter y Prettier
        \item Mypy y flake8
        \item GitHub Actions
        \item Test y virtualización: GH Actions, tox
        \item Uso de docker para test (usar comparativas - alpine) (Opcional)
        \item Despliegue en Vercel (usar comparativas - netlify)
    \end{itemize}
    \item Justificación de pydantic de python
    \item Justificación del micro-framework - FastAPI
    \item Justificación de front end framework - Vue3.js
    \item Justificación de TailwindCSS
    \item Justificación de Apexcharts
    \item Justificación de la base de datos MongoDB
    \item Implementación de sistema de Logs (Opcional)
    \item Prueba de prestaciones con uvicorn (Opcional)
\end{itemize}
\item Implementación:
\begin{itemize}
    \item Explicación del algoritmo que recoge trends de la API de Twitter
    \item Explicación del algoritmo que recoge noticias a partir de los trends - Comparar con la api y scrapping
    \item Explicación del algoritmo que recoge estadísticas del volumen de seguimiento (rango de tendencia)
    \item Explicación del algoritmo que recoge estadísticas de las palabras más usadas en la tendencia y YAKE
    \item Explicación del algoritmo que recoge estadísticas del sentimiento general de la tendencia (Vader) comparar con ML y texblob
\end{itemize}
    \item Conclusión
    \item Trabajos futuros y Bibliografía
\end{itemize}
Preguntas a tutores:
\begin{itemize}
    \item ¿El resumen debería llevar sangría?
    \item ¿Debería haber presupuesto en la parte del plan inicial?
\end{itemize}
Recordatorios:
\begin{itemize}
    \item Cambiar los HOLA
\end{itemize}

