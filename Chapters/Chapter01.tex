\chapter{Contexto}\label{ch:context}
%************************************************

\section{Motivación o justificación del proyecto}

\vspace{0.3cm}

El año 1989 fue clave en distintos aspectos, conocido por la disolución del Telón de Acero en Europa y la caída del Muro de Berlín, aunque sobre todo destacó por la apertura del Internet al público. Una herramienta, en un principio, al alcance de pocos que inicio su meteórico ascenso a ser uno de los mayores inventos. Hoy en día, capaz de conectar al mundo como muy pocos y que situaciones como la reciente pandemia COVID-19 no han hecho más que impulsar el intercambio digital de información entre personas.

\vspace{0.3cm}

Para podernos situarnos con más claridad, el estudio anual de IAB Spain nos indica que en el año 2022 el 93\% de las personas que residen en España son usuarios del Internet. Además de ese porcentaje, el 85\% (28,3 millones de españoles) utilizan las redes sociales. \cite{Studio-Aloha}

\vspace{0.3cm}

Estos datos nos dejan en evidencia el hecho de que el uso de estas herramientas, tales como el Internet y las redes sociales, van en aumento. Aunque no nos debería de asombrar, ya que la generalización y el uso del Internet y las redes sociales viene de mucho antes. En el año 2012 se realizó un estudio en el que se mostraba que las redes sociales pueden llegar a crear una dependencia aún más grande que el tabaco. \cite{Tabaco-Study}

\vspace{0.3cm}

Incluso muchos programadores de redes sociales han admitido que el diseño de tales aplicaciones es adictivo de manera deliberada. Un ejemplo de esto puede ser la creación del scroll infinito, donde se brinda gran cantidad de información de una manera continua. \cite{Social-Deliberately}

\vspace{0.3cm}

La solución más sencilla, pero drástica, puede llegar a ser abandonar las redes sociales, aunque nos surge el problema de no estar tan \textit{actualizados} como otras personas a nuestro alrededor. Otra solución es usar en una menor medida las redes sociales, aunque esto nos lleva de nuevo al problema de cuándo parar si la aplicación que usamos tiene scroll o contenido infinito.

\vspace{0.3cm}

Ante tal necesidad de querer estar actualizados y a la vez querer saber cuándo parar de consumir contenido, la solución que se propone en este texto es un consumo finito de contenido diario y lo más objetivo posible. Una aplicación web que aporte la información necesaria de las tendencias más destacables del día.

\section{Objetivos}

\subsection{Objetivo general}
El interés principal de este proyecto reside en la construcción de una aplicación que pueda explicar lo relevante en un momento dado, actualizando la información sin intervención humana y de manera completamente automática. A la hora de cumplir esta meta, se pretende llevar a cabo un estudio completo y óptimo sobre la planificación, el diseño, la estructura y arquitectura de la propia aplicación.

\vspace{0.3cm}

El propósito de la aplicación es que el usuario final pueda consumir contenido relevante de una manera eficaz y eficiente. Para poder proveer el contenido más relevante, podemos partir de la idea de que lo más destacable en internet es lo que más se repite o se comparte con otras personas. El concepto de lo más destacable en la red, suele recibir el nombre de \ac{TT}.

\vspace{0.3cm}

Por tanto, las tendencias o los \ac{TT} nos permiten llegar a conocer y saber en un momento dado lo que genera mayor conversación y debate entre la sociedad, pudiendo conocer los temas más actuales, ideas generales o multitud de voces que se hayan unido por una causa.

\vspace{0.3cm}

En consecuencia, se ha decido como óptimo que la información que se le brinda al usuario se basará en las tendencias, pudiendo dividirlas por el país y la fecha seleccionada por él. Al querer que el contenido no sea abrumante, las tendencias se presentarán en un \textit{top} o una clasificación de diez tendencias más populares ocurridas a lo largo del día. A partir de cada tendencia se deducirán datos, información útil, estadísticas, noticias y enlaces de interés que le puedan ser útiles al usuario.

\vspace{0.3cm}

La visualización de contenido que pretende tener esta página web, es un contenido presentado en forma de cartas o módulos, con un carácter informativo, finito y objetivo. Los datos deducidos o extraídos de la tendencia serán datos relevantes como la popularidad o el volumen de publicaciones que está teniendo la tendencia. La información útil que se pretende mostrar serán las palabras claves o más usadas relacionadas a la tendencia. También se pretende mostrar la opinión general de manera objetiva, esto se puede conseguir analizando los sentimientos de las publicaciones mediante analizadores léxicos. Por último, las noticias extraídas también albergarán contenido actual y relevante sobre dichas tendencias.

\vspace{0.3cm}

Para cumplir con dichas implementaciones, se deberá hacer un cuidadoso estudio sobre diferentes librerías, bibliotecas externas y algoritmos que se vayan a emplear en el proyecto. En la sección de las palabras claves se enfocará sobre el algoritmo de extracción \ac{YAKE} y en el análisis de sentimientos se centrará en la librería \ac{VADER}.

\vspace{0.3cm}

Por último, conviene mencionar que toda esta información recolectada deberá albergar un carácter sencillo y deberá ser fácil de consumir, alejándose de la posibilidad de ser abrumante o atosigadora, como la de las redes sociales.

\subsection{Objetivos específicos}

\begin{itemize}
\item Se realizará un estudio sobre el diseño, la estructura y arquitectura más óptima al proyecto. Al igual que las distintas herramientas usadas en este.
\item Se marcarán las pautas de unas buenas prácticas a la hora de realizar el desarrollo y las operaciones del proyecto.
\item Se estudiarán las ventajas de recolectar las tendencias provenientes de la API de Twitter y se compararán con otros métodos destacables.
\item La aplicación web debe de ser intuitiva y fácil de usar, el diseño debe ajustarse para el formato móvil.
\item El contenido que se le brinda al usuario debe de ser corto y finito, pero a la vez suficiente con el fin de informarse de manera adecuada.
\item Cada contenido proporcionado para el usuario se brindará en un diseño de carta y ocupará el mayor espacio de pantalla posible. De esta manera se aprovecharán al máximo la información disponible, siendo esta una noticia, un dato de una estadística o un enlace de interés.
\item Cada información recolectada dispondrá de un elemento gráfico y textual. La popularidad, las palabras más relevantes o los sentimientos generales se pueden expresar mediante una gráfica y también explicarse textualmente. Al igual que las noticias, cada noticia puede tener una imagen y una descripción.
\item El usuario podrá navegar por dichas tendencias y estar \textit{actualizado} en cada momento del día, además podrá ver tendencias de otros días o países. Siempre respetando que el contenido sea finito, es decir no atosigante, y relevante o actual.
\end{itemize}
